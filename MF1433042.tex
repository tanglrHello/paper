%% 使用 njuthesis 文档类生成南京大学学位论文的示例文档
%%
%% 作者:胡海星,starfish (at) gmail (dot) com
%% 项目主页: http://haixing-hu.github.io/nju-thesis/
%%
%% 本样例文档中用到了吕琦同学的博士论文的提高和部分内容,在此对他表示感谢。
%%
\documentclass[master, winfont]{njuthesis}
%% njuthesis 文档类的可选参数有:
%%   nobackinfo 取消封二页导师签名信息。注意,按照南大的规定,是需要签名页的。
%%   phd/master/bachelor 选择博士/硕士/学士论文

% 使用 blindtext 宏包自动生成章节文字
% 这仅仅是用于生成样例文档,正式论文中一般用不到该宏包
\usepackage[math]{blindtext}
\usepackage{algorithm}
\usepackage{algorithmic}
\usepackage{ulem}
\usepackage{multirow}
\usepackage{tikz}
\usepackage{array}
\usepackage{xfrac}
\usepackage{listings}
\usetikzlibrary{shapes,arrows}
%\setlength\titlebox{5cm}
\renewcommand{\algorithmicrequire}{ 输入:} %Use Input in the format of Algorithm
\renewcommand{\algorithmicensure}{ 输出:} %UseOutput in the format of Algorithm
\floatname{algorithm}{算法}
%%%%%%%%%%%%%%%%%%%%%%%%%%%%%%%%%%%%%%%%%%%%%%%%%%%%%%%%%%%%%%%%%%%%%%%%%%%%%%%
% 设置《国家图书馆封面》的内容,仅博士论文才需要填写

% 设置论文按照《中国图书资料分类法》的分类编号
%\classification{0175.2}
% 论文的密级。需按照GB/T 7156-2003标准进行设置。预定义的值包括:
% - \openlevel,表示公开级:此级别的文献可在国内外发行和交换。
% - \controllevel,表示限制级:此级别的文献内容不涉及国家秘密,但在一定时间内
%   限制其交流和使用范围。
% - \confidentiallevel,表示秘密级:此级别的文献内容涉及一般国家秘密。
% - \clasifiedlevel,表示机密级:此级别的文献内容涉及重要的国家秘密 。
% - \mostconfidentiallevel,表示绝密级:此级别的文献内容涉及最重要的国家秘密。
% 此属性可选,默认为\openlevel,即公开级。
%\securitylevel{\controllevel}
% 设置论文按照《国际十进分类法UDC》的分类编号
% 该编号可在下述网址查询:http://www.udcc.org/udcsummary/php/index.php?lang=chi
%\udc{004.72}
% 国家图书馆封面上的论文标题第一行,不可换行。此属性可选,默认值为通过\title设置的标题。
%\nlctitlea{数据中心}
% 国家图书馆封面上的论文标题第二行,不可换行。此属性可选,默认值为空白。
%\nlctitleb{网络模型研究}
% 国家图书馆封面上的论文标题第三行,不可换行。此属性可选,默认值为空白。
%\nlctitlec{}
% 导师的单位名称及地址
%\supervisorinfo{南京大学计算机科学与技术系~~南京市汉口路22号~~210093}
% 答辩委员会主席
%\chairman{张三丰~~教授}
% 第一位评阅人
%\reviewera{阳顶天~~教授}
% 第二位评阅人
%\reviewerb{张无忌~~副教授}
% 第三位评阅人
%\reviewerc{黄裳~~教授}
% 第四位评阅人
%\reviewerd{郭靖~~研究员}

%%%%%%%%%%%%%%%%%%%%%%%%%%%%%%%%%%%%%%%%%%%%%%%%%%%%%%%%%%%%%%%%%%%%%%%%%%%%%%%
% 设置论文的中文封面

% 论文标题,不可换行
\title{面向高考问答的地理试题文本分析和标注研究}
% 论文作者姓名
\author{汤莲瑞}
% 论文作者联系电话
\telphone{}
% 论文作者电子邮件地址
\email{tanglr@nlp.nju.edu.cn}
% 论文作者学生证号
\studentnum{MF1433042}
% 论文作者入学年份(年级)
\grade{2014}
% 导师姓名职称
\supervisor{戴新宇~副教授}
% 导师的联系电话
\supervisortelphone{}
% 论文作者的学科与专业方向
\major{计算机技术}
% 论文作者的研究方向
\researchfield{自然语言处理}
% 论文作者所在院系的中文名称
\department{计算机科学与技术系}
% 论文作者所在学校或机构的名称。此属性可选,默认值为``南京大学''。
\institute{南京大学}
% 论文的提交日期,需设置年、月、日。
\submitdate{2017年5月22日}
% 论文的答辩日期,需设置年、月、日。
\defenddate{2017年5月25日}
% 论文的定稿日期,需设置年、月、日。此属性可选,默认值为最后一次编译时的日期,精确到日。
\date{2017年5月22日}

%%%%%%%%%%%%%%%%%%%%%%%%%%%%%%%%%%%%%%%%%%%%%%%%%%%%%%%%%%%%%%%%%%%%%%%%%%%%%%%
% 设置论文的英文封面

% 论文的英文标题,不可换行
\englishtitle{Research of Text Analysis and Data Annotation for NCEE Geography Question-Answering}
% 论文作者姓名的拼音
%\englishauthor{}
\englishauthor{Lianrui Tang}
% 导师姓名职称的英文
%\englishsupervisor{}
\englishsupervisor{Associate Professor Xinyu Dai}
% 论文作者学科与专业的英文名
\englishmajor{Computer Technology}
%\englishmajor{Computer Technology}
% 论文作者所在院系的英文名称
\englishdepartment{Department of Computer Science and Technology}
% 论文作者所在学校或机构的英文名称。此属性可选,默认值为``Nanjing University''。
\englishinstitute{Nanjing University}
% 论文完成日期的英文形式,它将出现在英文封面下方。需设置年、月、日。日期格式使用美国的日期
% 格式,即``Month day, year'',其中``Month''为月份的英文名全称,首字母大写;``day''为
% 该月中日期的阿拉伯数字表示;``year''为年份的四位阿拉伯数字表示。此属性可选,默认值为最后
% 一次编译时的日期。
\englishdate{May 22, 2017}

%%%%%%%%%%%%%%%%%%%%%%%%%%%%%%%%%%%%%%%%%%%%%%%%%%%%%%%%%%%%%%%%%%%%%%%%%%%%%%%
% 设置论文的中文摘要

% 设置中文摘要页面的论文标题及副标题的第一行。
% 此属性可选,其默认值为使用|\title|命令所设置的论文标题
% \abstracttitlea{数据中心网络模型研究}
% 设置中文摘要页面的论文标题及副标题的第二行。
% 此属性可选,其默认值为空白
% \abstracttitleb{}

%%%%%%%%%%%%%%%%%%%%%%%%%%%%%%%%%%%%%%%%%%%%%%%%%%%%%%%%%%%%%%%%%%%%%%%%%%%%%%%
% 设置论文的英文摘要

% 设置英文摘要页面的论文标题及副标题的第一行。
% 此属性可选,其默认值为使用|\englishtitle|命令所设置的论文标题
\englishabstracttitlea{Research of Text Analysis and Data Annotation}
% 设置英文摘要页面的论文标题及副标题的第二行。
% 此属性可选,其默认值为空白
\englishabstracttitleb{for NCEE Geography Question-Answering}

%%%%%%%%%%%%%%%%%%%%%%%%%%%%%%%%%%%%%%%%%%%%%%%%%%%%%%%%%%%%%%%%%%%%%%%%%%%%%%%
\begin{document}

%%%%%%%%%%%%%%%%%%%%%%%%%%%%%%%%%%%%%%%%%%%%%%%%%%%%%%%%%%%%%%%%%%%%%%%%%%%%%%%

% 制作国家图书馆封面(博士学位论文才需要)
%\makenlctitle
% 制作中文封面
\maketitle
% 制作英文封面
\makeenglishtitle


%%%%%%%%%%%%%%%%%%%%%%%%%%%%%%%%%%%%%%%%%%%%%%%%%%%%%%%%%%%%%%%%%%%%%%%%%%%%%%%
% 开始前言部分
\frontmatter

%%%%%%%%%%%%%%%%%%%%%%%%%%%%%%%%%%%%%%%%%%%%%%%%%%%%%%%%%%%%%%%%%%%%%%%%%%%%%%%
% 论文的中文摘要
\begin{abstract}
人工智能技术正在飞速改变这个世界。在自然语言领域,围绕着自动问答系统(Question Answering, QA)开展了越来越多的研究。高效、智能的问答系统,致力于为用户提供更直接更优质的答案,可以从大量的知识储备中自动进行检索、推理,从而将用户从这些处理中解放出来。2011年,IBM的Watson问答机器人参加问答类综艺节目“Jeopardy!”,并战胜了人类顶尖选手赢得冠军,自动问答系统再一次吸引了世人的眼光。

从某种程度上来说,高考作为中国大多数中学生最重要的考试,可以看做是一种高水平的问答过程。本文的项目背景是面向中国高考地理试题的问答系统,并侧重于对选择题的解答。在解决高考自动问答的过程中,我们面临很多与传统问答系统不同的挑战:首先高考题的问答形式与传统自动问答系统存在明显区别;其次,高考题的灵活性远高于传统问答系统中的问题,这意味着我们很难从现成的文本中直接匹配、抽取得到答案。

作为自动问答的第一步,问题理解的作用十分重要,这也是本文的工作重点。本文中将选择题题面和一个选项拼接成的完整句子作为分析的对象。对文本的理解可以分为两种:一是对句子间的篇章关系分析,二是对句子内部的语义关系理解。因此我们从两个方面来研究对于地理试题的理解问题:一方面是问题中子句间关系分类,另一方面是尝试使用AMR(Abstract Meaning Representation)对试题文本进行深层语义分析。

本文中的句子拆分工作,是针对地理选择题的特点,提出了利用逗号对选择题的选项进行可能的拆分,将较长的原句转换成语义等价的多个简单句,从而简化后续处理步骤的输入,提高后续步骤的处理能力。在这项工作中,我们使用了最大熵分类器和基于规则的启发式方法,通过两个步骤来实现句子拆分:首先识别选项中的逗号是否可以作为一个分割点,然后再识别句子的从句或并列结构的公共前缀边界。

AMR是一种具有较为强大表达能力的新型语义表示方法,它可以将一句话的语义用单根的、有向的连通图表示出来,更强调句子的抽象语义,而非具象的语法表达方式。但是由于围绕AMR的研究才刚刚起步,目前已有的AMR自动分析效果还有很大待提升的空间。中文AMR的标注语料规模较小,将AMR应用到中文的研究几乎还是空白。本文在AMR方面的工作主要是对现有AMR分析算法进行一些实验分析,并首次验证AMR标注体系及自动解析算法在中文上的性能。针对地理试题,我们标注了一个小样本的AMR语料,并用现有算法来验证AMR在特定领域文本上的处理能力。

为了支撑上述两项问题理解的研究工作,我们还构建了一个地理试题标注工具,并通过这个工具建立一个高质量的地理试题语料库。除了可以标注句子分割和AMR这两种信息,该工具同时支持标注分词、词性、命名实体、地理术语、试题模板表示、成分句法等各项数据。

\keywords{问题理解;句子拆分;语义分析;AMR;地理文本;标注工具;}
\end{abstract}

%%%%%%%%%%%%%%%%%%%%%%%%%%%%%%%%%%%%%%%%%%%%%%%%%%%%%%%%%%%%%%%%%%%%%%%%%%%%%%%
% 论文的英文摘要
\begin{englishabstract}
Artificial intelligence is changing the world rapidly. In the field of natural language process, more and more researches on automatic question-answering have been carried out. Highly efficient and intelligent QA(Question Answering) systems aim to provide more direct and precise answers to users. They can retrieve information from large-scale knowledge base and make deductions automatically. Therefore, they free users from searching and filtering texts from the large quantity of information, as well as finally extracting the answers. In 2011, the QA robot Watson from IBM took part in a quiz show named Jeopardy!, beat the top human players and became the champion. Once again, QA system attracted the attention of the world.

To some extent, the National College Entrance Examination(NCEE) is the most important test for the majority of Chinese high school students, which can be seemed as a high-level question-and-answer process. The background of this paper is a question answering system focused on the geography questions of the NCEE. And our work pays more attention to the multiple-choice questions. In the process of developing this QA system, we are faced with many challenges, unlike those we meet in traditional QA systems. Firstly, the style of questions is quite different. Secondly, the questions are much more flexible, which means we can hardly match the question with the original texts in the knowledge base directly.

As the first step of automatic question-answering, question understanding plays a key role for the whole system, which is the key point of this paper as well. Our strategy is as follows: for each choice, we join it with the question to get a complete sentence, which will be used as a basic subject for analysis. There are two common types of comprehension for text: one is the discourse analysis between clauses, and another is deep semantic parsing for the sentences. Thus we work on improving the understanding of the sentences through the following two aspects: 1. Classify the relationship of the parts separated by comma in long and complicated sentences; 2. Using AMR to do deep semantic parsing. 

For splitting sentence, we propose a method of splitting by the commas in the choices, according to the feature of multiple-choice questions. Then possibly we may transform long original sentences into several semantically equivalent shorter sentences, thus promote the performance of the following processing stages. In this part, we put forward a two-stage method, using MaxEntropy classifier and a rule-based method in each stage respectively. First, we recognize whether the comma in the choice could be seen as a splitter. Second, find out the right border of the common prefix for the coordinate structure in the sentence.

AMR(Abstract Meaning Representation) is powerful semantic representation method, which is newly proposed. It can represent the semantic of a sentence as a rooted, directed and connected graph. It focuses more on the abstract semantic in the sentence, instead of the superficial syntax style. However, the research on AMR has just started, so the state-of-the-art automatic AMR parsing algorithms are still not satisfactory enough. Chinese AMR corpus is relatively small at the moment. As far as we know, no research and application has applied AMR to Chinese corpus yet. Our work is based on an English AMR parsing tool. In this paper, we modify this tool to process Chinese, and verify the performance of this algorithm on Chinese corpus. For geography questions, we get a small AMR annotation corpus and run this algorithm on it, too.

In order to support the two parts of work mentioned above, we developed a annotation tool for the NCEE geography questions. With the help of this tool, we build a high-quality corpus on geography questions. In addition to the sentence splitting and AMR annotation, this tool also supports segmentation, part-of-speech, named entity, geography terms, question template representation and syntactical parsing annotation.

% 英文关键词。关键词之间用英文半角逗号隔开,末尾无符号。
\englishkeywords{Question Comprehension, Semantic Parsing, AMR, Geographic Text, Annotation Tool, Sentence Simplification}
\end{englishabstract}

%%%%%%%%%%%%%%%%%%%%%%%%%%%%%%%%%%%%%%%%%%%%%%%%%%%%%%%%%%%%%%%%%%%%%%%%%%%%%%%
% 论文的前言,应放在目录之前,中英文摘要之后
%
%\begin{preface}
%\section{研究背景}
%想
%\vspace{1cm}
%\begin{flushright}
%程善伯\\
%2013年夏于南京大学
%\end{flushright}
%
%\end{preface}

%%%%%%%%%%%%%%%%%%%%%%%%%%%%%%%%%%%%%%%%%%%%%%%%%%%%%%%%%%%%%%%%%%%%%%%%%%%%%%%
% 生成论文目次
\tableofcontents

%%%%%%%%%%%%%%%%%%%%%%%%%%%%%%%%%%%%%%%%%%%%%%%%%%%%%%%%%%%%%%%%%%%%%%%%%%%%%%%
% 生成插图清单。如无需插图清单则可注释掉下述语句。
\listoffigures

%%%%%%%%%%%%%%%%%%%%%%%%%%%%%%%%%%%%%%%%%%%%%%%%%%%%%%%%%%%%%%%%%%%%%%%%%%%%%%%
% 生成附表清单。如无需附表清单则可注释掉下述语句。
\listoftables

%%%%%%%%%%%%%%%%%%%%%%%%%%%%%%%%%%%%%%%%%%%%%%%%%%%%%%%%%%%%%%%%%%%%%%%%%%%%%%%
% 开始正文部分
\mainmatter

%%%%%%%%%%%%%%%%%%%%%%%%%%%%%%%%%%%%%%%%%%%%%%%%%%%%%%%%%%%%%%%%%%%%%%%%%%%%%%%
% 学位论文的正文应以《绪论》作为第一章
\chapter{绪论}\label{chapter_introduction}
\section{研究背景}
在人工智能技术日新月异的今天,人们对人工智能技术寄予了越来越多的期待。在各个领域,人们都在不断试图突破人工智能目前的极限。在自动问答领域,已经有很多商业化的系统为企业提供高效的解决方案,为用户提供更加快捷、准确的服务。在自动问答出现以前,我们获取知识的方式通常是在搜索引擎中搜索关键字,在得到的网页文本中一个个去搜寻我们想要的答案。问答系统的出现是对搜索引擎功能的一次升级。问答系统希望不但能够从海量数据中找到与用户问题相关的文本,还能够从文本中直接准确地找出答案,免去使用者自己去从搜索结果中进一步寻找答案的过程。

通常我们所说的问答系统可以针对一个自然语言的问句,在知识库中找到相关的支持文本,然后可能通过一些简单的推理,抽取出可能的答案,再对所有答案进行综合打分,并将最终的答案返回给用户。waston系统是这类问答系统的一个变种,输入是一个陈述句,但是句中会用一个代词来代替询问的命名实体,waston所做的事情就是首先识别出哪个代词是需要消解出来的,然后进行上述的问答系统的流程\cite{Ferrucci2010}。

从某种角度来说,考试的过程就是一种问答过程,而高考作为中国学生进入高等教育的关键考试,其试题更具有难度和代表性,高考是对考试者的知识积累、推理能力、判断能力的一种综合考察。为了探索问答系统的潜力,基于863项目《开放域知识集成、推理与检索关键技术及系统》,我们对地理试题的自动解答进行了研究。在高考地理试题中,选择题是一类重要的题型。不同于传统的问答系统,选择题不是一个有明确疑问词的疑问句,也不像waston那样去消解一个句子中未知的代词。选择题更像是对四个选项的陈述做出判断的判断题。并且很难直接从课本或者其他文本中得到相关文本,再通过匹配来判断一个句子是否正确,而是需要根据上下文中的时间、地点、假设等等,综合相关的知识点,经过复杂的推理和计算才能够得到正确答案,因此和传统的问答系统存在很大区别。

在试题自动解答的过程中,对问题的理解是一个关键步骤。这一步包括对问题做各种基础的自然语言处理,得到一些基本的分析结果。对每一项基础分析任务,我们需要针对地理领域试题的特点,做出一些针对性的调整,提高通用工具对地理试题的处理能力。在本文中,主要从句子拆分简化和AMR语义表示解析两方面来帮助系统对试题的理解。对于地理选择题的特点,我们提出了基于选项中的逗号对句子进行拆分简化的方法。另外,对于句子语义的理解,除了目前比较常见的语义角色标注等语义分析方法,我们尝试使用了近些年新提出的AMR方法,这个表示体系对于句子语义有更强的潜在表示能力,也是一个值得探索的新方向。本文基于目前已有的AMR算法,对AMR在中英文语义表示和自动分析方面做了一些实验对比分析,探索AMR目前可以达到的水准,并在一个小的地理领域试题数据上进行了实验,希望能够为后面的问题理解工作探索一个新思路。

\section{问题理解研究现状}
输入的问题为疑问句类型的问答系统,一般有三个主要组成部分,即问题理解、信息检索、答案抽取。问题理解阶段的任务是,充分理解用户的提问意图,即用户想问的是什么。一般来说,问题理解模块首先要对问句进行预处理;然后,需要进行问题分类和问题扩展\cite{Cao2005QA},以及对问题做语义分析。

在问句预处理阶段,主要包括分词、词性标注、句法分析等工作。这些基础工作的研究已经相对成熟,一般不作为问句分析中的重点进行研究。问题语义分析和一般句子的语义分析总的来说也是交叉的,可以视为一类特殊句子的理解,语义分析的

对于问题分类,现有的分类体系主要包括三种:基于问句语义信息、基于答案类型、基于混合信息\cite{ZhangNing2016}。问句分类的作用是使候选答案空间有效减小,提高检索的效率,还可以相应地制定答案抽取的策略。

问句分类早期主要使用基于规则的方法,通过人工分析句法结构来提取规则,主观性比较强,并且专家的分类决策对分类体系有很大的影响,灵活性较差。随后出现的基于统计学习的方法表现出良好的分类效果,但分类精度会受到句法分析等上游处理任务精度的影响。之后,有研究者将两种方法结合应用到问句分类中,取得了较好的分类效果。

张宇等人\cite{ZY2004}使用词语之间的无关性,将贝叶斯模型来应用到中文问句的分类上。该方法实现起来比较容易,但侧重词频信息,相对而言忽略了句法结构及语义信息。文勖等人\cite{Wen2006}在句法分析的结果中,提取出疑问词、问题主干信息、其他附属成分作为特征。孙景广等人\cite{Sun2007}将知网中的语义信息作为特征,使用最大熵模型来做问句的分类,结果表明该做法可以明显提升分类的准确度。牛彦清等人\cite{Niu2012}将特征分为六种,包括:问题疑问词与核心关键词的主要义原、核心关键词的首义原、问句主谓宾的主要义原、命名实体、名词单复数(问的是一个实体还是几个实体)。然后使用支持向量机进行分类,组合不同特征后发现,使用词义消歧得到的主要义原对分类精度有明显影响,同时还能显著较少特征向量的维度。杨思春等人\cite{YangSC2014}对于特征组合的问题,提出基于重要性及差异性的特征组合。王小林等人\cite{WangXL2014}提出将增量式贝叶斯思想用于问句分类,来解决分类模型不能随用户的需求而改变的问题,该方法用遗传算法选取最优的特征子集,以此优化分类器,使其能够在学习时调整参数。

问题扩展的用处是提高信息检索步骤的查全率。因为原始的问句一般都比较单一,不可能包含查找相关文档的所有词语。对问句进行扩展后,可以检索到更多与问题语义相关的文档,提高了找到答案的可能性。曹志娟等人\cite{Cao2005QA}提出了问句重写、关键词扩展这两种方式来进行问题扩展。问句重写即是为每一种类型制定重写规则,以不同的表达方式来表达同样的问题。而关键词扩展是针对同义词现象提出的,问句的关键词通常都会有同义词存在,有时包含答案的段落中包含的其实的关键词的同义词,所以在进行信息检索之前,对关键词进行同义词的扩展可以找到更多的相关文本。刘茂福等人\cite{Liu2012QA}使用维基百科中与问题相关的页面,找到其中与问题最相关的段落,对查询进行扩展。

\section{论文的主要工作}
本文工作主要是关于地理试题文本的问题理解,具体是从下面三个主要方面开展工作:

其一,针对选择题选项的特点,我们提出对部分含有逗号的选项进行句子简化拆分。我们发现地理选择题中有14\%的选项中包含一个或一个以上的逗号。在这些包含逗号的选项中,有71.7\%的选项可以通过在某处寻找一个边界,将句子分割成公共部分和非公共部分,然后组合得到多个可以分别判断正误的句子。虽然这部分可拆分的选项在我们标注的所有数据中只占10.1\%,但是我们发现,在后续试题语义模板化处理的过程中,对于这类句子的处理难度较大,较严重影响地影响了自动模板化的性能,因此该工作对提高自动模板化的性能有显著意义。

其二,探索AMR在中英文语义分析上的效果,及其在地理试题上的应用效果。AMR是一种新型的语义表示方式。在此之前,我们在高考问答系统中使用的主要语义分析方法,是将试题文本通过其句法结构和词性等特征,转换为一个我们制定的地理试题模板体系中的模板表示,每个模板根据其定义包括模板类型和语义槽,每个选择题和问答题的核心问题部分都可以转换成一个语义模板的表达。经调研发现,AMR的表示体系中有很多类似的语义结构,如果针对地理文本对某些关键的实体和关系进行AMR的解析,在理想的效果下可以方便地转换为我们所需要的模板表达。所以我们探索了AMR在中文上的自动分析能力,以便未来将其应用到问题理解任务中。

其三,为了支撑上述两项围绕问题理解的研究,我们需要构建一个地理试题的语料库,为此开发了地理试题标注系统。由于研究开始时缺乏足够的有标注地理文本,我们针对地理试题的结构特点,包括文本外部组织结构和文本内部的语法表达方式,设计并开发了一个标注系统,该系统支持特定格式的试卷导入,同时支持选择题和主观题试卷的标注,能够保留试卷原有的结构信息(比如高考题常有一些几道连续的题目共享的背景知识,主观题中题目的嵌套,地理试题的主选项和小选项等等),支持对包括分词、词性、命名实体、术语、语义模板表示、成分句法分析、选择题主次文本、AMR的标注。可以通过关键字、试卷名、某一项数据的标注状态、模板类型等多种方式进行试题检索,可以导出所有标注内容,提高了地理试题的标注效率和标注数据的检索和使用效率。

通过上述工作,我们实现了一个地理题拆分方法,在尽可能提高不可拆分句的召回率的情况下,找出了相当一部分可拆分句,为模板语义化等后续工作提供了更多的信息。此外,通过对中英文AMR语料进行自动解析实验,展示了当前的工具在中文上的处理尚有欠缺,通过多种实验给出了AMR应用于中文的现状。我们还建立了一个包含多项标注结果的地理试题语料,支持上述两项工作及其他更多问答系统项目中的任务。

\section{论文的组织}
本文内容的组织如下:

第一章主要介绍本文的研究背景,以及论文主要工作内容,并简述了问答系统中问题理解研究的现状。

第二章介绍本文提出的基于逗号对选择题选项进行拆分简化的工作。首先会介绍基于逗号的句子分解的研究现状,然后阐述本文工作的内容。包括选项拆分的动机,具体如何对选择题选项文本进行分割。分割主要分为两个步骤,第一步是判断句子是否可拆分,第二步是寻找到拆分的边界,类似补全句子成分。本章描述了我们选取的文本特征和算法,给出实验结果及错误分析。

第三章介绍AMR在中英文语义理解上的工作。首先介绍目前AMR的研究现状和背景知识,包括标注体系及规范,语料建设情况,自动解析算法等内容。对于这个较新的任务,我们使用了目前性能比较领先的一种图算法,对多种中英文语料设计了多种实验,对标注数据和算法流程进行了阐述,并在少量地理题标注数据上做了实验,为后续AMR相关的工作提供参考。本章总结了JAMR对于中文AMR解析目前存在的一些缺陷。

第四章主要介绍地理试题标注系统的相关内容,包括系统架构、功能设计、数据库设计、交互方式等各个方面。

第五章主要是对本文工作的总结以及对未来工作的展望。

\chapter{背景知识}
\section{引言}
对于自然语言理解,有两大块研究方向,一是句子之间的篇章关系理解,一是句子内部的语义理解。篇章关系指出了子句之间的语义关系,语义理解则指出了句子里词语之间的语义关系。本文的两项主要工作中,第一项是根据逗号对句子进行可能的切分,实际上是对逗号类型的识别,也是对逗号两侧是否构成两个子句、两个子句之间是否为特定的篇章关系的识别;第二项工作则属于一种语义理解的新方法。因此本章背景知识将从篇章分析和语义理解两个角度来介绍。

\section{篇章分析}
汉语篇章分析的起步较晚,目前的研究较少,相关的语料也不多。目前比较有代表性的、完整的研究是李艳翠等人\cite{liyancui2015}对中文篇章结构体系表示和资源建设上的工作。该工作中,从子句的定义的判定、篇章连接词、篇章结构的关系、篇章主次判定等四个方面,完整地对汉语篇章进行了分析。其中,子句判断及相关的逗号分类体系、篇章关系类别体系与我们的工作比较相关,本节主要从这些方面来介绍。

\subsection{子句判定}
对于篇章分析中的篇章基本单位如何定义,有许多不同的观点。在李艳翠的工作中,则作如下定义:结构上,子句至少包含一个谓语成分,至少表达一个命题;功能上,子句对外不作为其它子句结构的语法成分,子句和子句间发生命题关系;形式上,子句间一定有标点(通常是逗号、分号、句号等)分割\cite{liyancui2015}。

所以,子句通常有两种情况:子句是一个单句,单句能够表达一个相对完整的意思,全句主干只有一个结构中心,每个结构中心只能有一个主谓结构或非主谓结构;子句是复句中的分句,复句之间一般以标点符号来停顿,通常有逗号、分号或冒号\cite{liyancui2015}。

例如,“a古老的京杭大运河如今不仅在贯通南北运输方面发挥重要作用,|b而且带动起一条欣欣向荣的工业走廊,||c形成了大运河经济带。”这个例子中有三个子句,b和c之间是因果关系,连接词是隐含的;a和bc的组合之间是递进关系,连接词是“不仅……而且……”。

再比如,“浦东开发开放是一项振兴上海,建设现代化经济、贸易、金融中心的跨世纪工程”,虽然有一个逗号,但它只能作为一个子句而不是两个。

汉语书面语中,标点符号分为标号(如引号)和点号(如句号)两类。可以用作子句边界的是点号,其中句号、问好、叹号、分号一定表示子句边界;顿号一定不是子句边界;逗号和冒号可能是,也可能不是子句边界。\cite{liyancui2015}在所有子句边界标点统计中,冒号仅占1.5\%,而逗号占了61.1\%,所以对逗号类型的判断十分重要。

\subsection{逗号分类体系}
\label{commaclssify}
李艳翠对逗号的分类参考了Yang Y.Q.\cite{Yang2012Chinese}的研究工作,介绍他们在研究中文篇章关系时整理出的逗号在句子中的功能分类,这个分类与我们对逗号功能的判断和应用比较吻合。在第\ref{geodatafeature}节中,我们会详细介绍地理试题文本中的逗号使用情况,并描述地理试题的分类的标准与上述的逗号分类体系之间的关联。

该工作将逗号的使用方法划分为7类,首先把逗号的使用方法在总体上分为两大类:一类是逗号连接的两个句子片段之间存在关系,即逗号是子句边界;另一类是两个句子片段之间不存在关系,不能视作是子句或者是篇章的边界。两个子句之间存在的关系又可以分为并列关系、从属关系两种:并列关系可分为三种类型(SB、IP\_COORD、VP\_COORD),从属关系也可分为三种类型(ADJ、COMP、SBJ),图\ref{comma_types}展示了这个分类体系。

\begin {figure}
	\centering
	\includegraphics[width=0.7\linewidth]{{comma}.jpeg}
	\caption{Yang的逗号分类体系}
	\label{comma_types}
\end{figure}

下面我们详细介绍一下每种类型的逗号的具体作用。

1)SB(Sentence Boundary):可以起到分割句子边界的作用的逗号。这种逗号要求其左右的子句都是IP结构,父节点为根节点。如图\ref{sb_comma}所示,该例中的两个逗号都是SB类型。

\begin {figure}
	\centering
	\includegraphics[width=0.8\linewidth]{{sb_comma}.jpeg}
	\caption{SB类型的逗号}
	\label{sb_comma}
\end{figure}

2)IP\_COORD(IP Coordination):用于分割父节点为非根节点的并列IP结构的逗号。如图\ref{ip_coord}所示,其中P4和P5属于该类型。

\begin {figure}
	\centering
	\includegraphics[width=0.7\linewidth]{{ip_coord}.jpeg}
	\caption{IP\_COORD类型的逗号}
	\label{ip_coord}
\end{figure}

3)VP\_COORD(VP Coordination):分割共享同一个主语的并列动宾短语的逗号。如图\ref{vp_coord}所示,其中P6属于该类型。

\begin {figure}
	\centering
	\includegraphics[width=0.6\linewidth]{{vp_coord}.jpeg}
	\caption{VP\_COORD类型的逗号}
	\label{vp_coord}
\end{figure}

4)ADJ(Adjunction):分割附属从句与主句的逗号。附属从句指在句子中担当某种句子成分的主属结构,虽然从句部分的句子结构是完整的,但它不能脱离主句部分独立完成地表达语义。附属从句往往是状语从句,通常有条件状语、目的状语、原因状语、伴随状语、方式状语等等。例如“为了在运行机制上与保护区相配套,宁波保护区率先在中国实施了企业依法注册直接登记制的试行一站式管理。”中的逗号前是一个目的状语。

5)COMP(Complementation)分割句子谓语与宾语的逗号。对宾语部分较长的复杂句子,通常会在谓语之后出现逗号,表示停顿,用来舒缓语气,通常在“表示”、“指出”、“介绍”等提示性动词之后都会出现逗号。例如“钱其琛表示,我们对香港的前景始终是充满信心的。”中的逗号。

6)SBJ(Subjective):分割句子主语和谓语的逗号。在句法结构上,其左兄弟节点为IP-SBJ或者NP-SBJ结构,而右兄弟节点为VP结构。例如“出口快速增长,成为推动经济增长的重要力量。”中的逗号。

7)OTHER:其他用法,在分类体系中对应逗号前后两个部分不存在关系的情况。

\subsection{连接词}
连接词是指将篇章单位连接起来,并且可以表示相应的连接关系语义的词语\cite{liyancui2015}。判断连接词的标准是看其联系的成分是否为子句,以及这个词能不能提示所联系篇章单位间的语义关系。例如“不仅……而且……”、“因此”等都可以作为连接词。

连接词的形式可以分为独用连接词、关联词、合用连接词和其他类型。独用连接词是一个单独的词语作为连接词,例如“并且”、“所以”;关联词指成对出现的连接词,如“既……又……”、“不但……而且……”;合用连接词是指几个词共同组成的连接词,例如“同时还”。

有时候,子句篇章之间会有显式的连接词,表明他们之间的篇章关系,但还有很多情况下,子句之间没有显式的连接词,但可以根据语义来判断两个子句之间的关系类别,然后为其添加连接词。

一般情况下,一个连接词表示一种篇章关系,但也存在一些连接词存在歧义或者多义。例如“于是”可以表示顺承关系,也可以表示因果关系。

\subsection{篇章关系分类体系}
篇章关系的类别实际上就是连接篇章的连接词的类型(包括添加的隐式连接词),主要可以分为四大类:因果类、转折类、并列类、解说类。每一类又可以细分为不同的类型,总共有17个小类,每个类型都包含若干连接词,有的连接词可以属于不同关系类型。关系分类体系结构如图\ref{relationtypes}所示。

\begin {figure}
	\centering
	\includegraphics[width=\linewidth]{{relationtypes}.jpeg}
	\caption{基于连接词的关系分类}
	\label{relationtypes}
\end{figure}

本文工作主要涉及并列篇章关系的识别,并列关系的并列项是叙述相关的几件事情或同一事物的几个方面,相关是在意义上并存、平行或对立。并列项可以是词、短语,也可以是句子。并列项可以互换位置,并列项逻辑上不互相依赖。

\section{语义理解}
语义理解分为浅层语义理解和深层语义理解两类。浅层语义理解的实现方式包括语义角色标注,深层的语义理解也有很多不同的做法,例如语义依存分析,AMR等。在本节中,会给出AMR的概念,关于AMR更详细的背景知识将在第\ref{amrback}节AMR背景知识中给出。

\subsection{语义角色标注}
语义角色标注(semantic role labeling, SRL)任务是找出句子中的所有谓词,并找出每个谓词在句中所有的论元,并标明论元的语义角色。一个谓词与其对应的论元集合形成一个命题,一个句子可能包含多个命题。例如“[委员会 Agent][明天 Tmp]将要[通过 V][此议案 Passive]”,其中“通过”是谓语动词,“委员会”、“此议案”、“明天”分别是其施事、受事和动作发生的时间\cite{LiuHJ2007}。

命题中的论元有两个性质:首先论元之间不会相互重叠;其次一个论元可能会由成几个不连续的短语组成,例如“[\{A1\} The apple], said John, [\{C-A1\} is on the table]”,这里两组方括号里面的内容共同组成了“said”的一个论元A1\cite{Carreras2004Introduction}。

语义角色标注有成熟的语料资源,英文是基于PTB(Peen TreeBank)中的华尔街杂志语料部分标注的PropBank(PropBank)\cite{Palmer2006The},中文则是基于CTB(Chinese TreeBank)标注的CPB(Chinese Proposition Bank)语料\cite{Xue2003Annotating}。谓词一般是一个动词,但也可以是名词。CPB语料标注的是动词性谓词及其论元,中文还有一个语义角色标注的语料NomBank(Nominal Bank)标注的则是名词性的谓词及其论元。

语义角色标注通常被看做分类问题,例如在谓词识别中,将词语分类为谓词和非谓词;在论元识别中,将词语分类为具体论元类型和非论元。目前的研究大多基于有监督的机器学习方法,例如支持向量机(Support Vector Machine, SVM)\cite{Pradhan2005Support}、最大熵模型(Maximum Entropy, ME)\cite{Kwon2004Senseval}等等。刘怀军等人的工作\cite{LiuHJ2007}指出,对于中文语义角色标注,单纯靠机器学习算法的改进,很难有质的提升。他们认为,丰富有效的特征对语义角色标注来说更为重要。很多语义角色标注工作也是围绕着如何使用新的特征展开的。根据实验结果,名词性谓词比动词性谓词的语义角色标注难度更高。

\subsection{语义依存分析}
 语义依存分析建立在依存理论的基础上,是一种深层的语义分析理论。它能提取句子中所有修饰词与核心词间对应的语义关系,且修饰词与核心词覆盖了句子中所有的词。语义依存分析是处理词级别、短语级别、从句级别以及句子级别的语义结构的过程,是面向整个句子的,而不仅是句子中主要谓词与其论元之间的语义关系,还含有非主要谓词包含的语义信息,如数量、属性、频率等等\cite{GuoJiang2011}。
 
 对于句子“一九九八年 美国 康奈尔 大学 研究生 莫里斯 写了 一个 程序”,图\ref{semdep}给出了它的语义依存分析结构(箭头方向为核心节点指向修饰节点)。
 
\begin {figure}[!htbp]
	\centering
	\includegraphics[width=\linewidth]{{semdep}.jpeg}
	\caption{语义依存结构}
	\label{semdep}
\end{figure}
 
语义依存关系分为三类,分别是主要语义角色(每一种语义角色对应存在一个嵌套关系和反关系)、事件关系(描述两个事件间的关系)、语义依附标记(标记说话者语气等依附性信息)。

语义依存语料库的建设包括两个过程:一是通过依存语法找出句子的依存结构,即所有的依存弧;二是确定语义关系集,并据此为所有的依存弧标上正确的语义依存关系。自动的语义依存分析通常也是分为这两步。

\subsection{AMR语义表示体系}
\label{subsection:amrback}
AMR是一种有向的、简单的、类似树结构的(树出现环变成图的比例比较小)、边和节点有标记的图\cite{Banarescu2013Abstract},可以将句子中的语义概念和概念之间的语义关系以节点和边的形式表现出来。

\begin {figure}[!htbp]
	\centering
	\includegraphics[width=0.5\linewidth]{{graph_amr}.png}
	\caption{AMR图表示}
	\label{amrgraph}
\end{figure}

AMR有多种表现形式,根据定义可以以一个图的方式呈现出来,如图\ref{amrgraph}。在图表示方法中,没有入边的节点就是根节点,在该图中就是“want-01”对应的节点。这张图是对“The boy wants the girl to believe him”这句话的AMR表示。根节点“want-01”中的后缀“-01”指定了这句中want这个谓词在Propbank体系中对应的语义框架,在这个语义框架下谓词有相应的多个论元,例如arg0和arg1这两个论元。

这句话中有一个“want”动作,它的arg0(谁在want)是“boy”,它的arg1(want的是什么)是一个“believe”的动作;“believe”动作的arg0(谁在believe)是“girl”,它的arg1(believe什么)是“boy”。所以在这里“boy”有两个作用,一是作为“want-01”的arg0,一是作为“belive-01”的arg1,这个节点有两条入边,因此AMR图上出现了一个环,这在AMR中也叫做“重入”(reentrancy)。也正是因为“重入”现象的存在,使得AMR是一种图表示方法,而不是树表示方法。

AMR还可以以文本方式表达出来:

\begin{lstlisting}[language=C]
(w / want-01 
   	:ARG0 (b / boy)
	:ARG1 (b2 / believe-01
		:ARG0 (g / girl)
		:ARG1 b))
\end{lstlisting}
 

第一行即AMR的根节点,冒号后面的标签表示节点之间边的标签,也就是节点之间的关系。每个节点由变量名(斜杠前的部分)和概念(斜杠后的部分)组成(除了多次出现的同一个节点),如果一个节点出现多次,则通过让该节点的变量名多次出现来表示,如该例中的节点b。

AMR也可以用逻辑三元组的来表示,上面的例子可以表示成:
\begin{lstlisting}[language=C]
   instance(w, want-01) ^      /* w 是want的实例 */ 
   instance(b, boy) ^          /* b 是boy的实例 */ 
   instance(b2, believe-01) ^  /* b2 是believing的实例 */ 
   instance(g, girl) ^         /* g 是girl的实例 */ 
   ARG0(w, b) ^                /* b 是want的发起者 */ 
   ARG1(w, b2) ^               /* b2 是被want的东西 */ 
   ARG0(b2, g) ^               /* g 是believe的人 */ 
   ARG1(b2, b)                 /* b 是被believe的人 */
\end{lstlisting}

上面例子中的这句话在英文中还可以有很多种不同的表示方法,例如:
\begin{lstlisting}[language=C]
   - The boy wants to be believed by the girl.
   - The boy has a desire to be believed by the girl.
   - The boy’s desire is for the girl to believe him.
   - The boy is desirous of the girl believing him.
\end{lstlisting}

概念want-01可以在句子中用动词want、名词desire或者形容词desirous来表达。

再比如下面这个AMR表示:
\begin{lstlisting}[language=C]
   (p / permit-01
       :ARG1 (g / go-02
           :ARG0 (b / boy))
       :polarity ))
\end{lstlisting}

也有很多种英文表达方式:
\begin{lstlisting}[language=C]
   - The boy may not go.
   - The boy is not permitted to go.
   - It is not permissible for the boy to go.
   - The boy does not have permission to go.
\end{lstlisting}

由此可以看出,AMR更加强调句子表达的逻辑语义,并将表层的语法表达方式抽象出去。

AMR是一种有向图表示,其中图的边和节点的关系应满足一些限制条件\cite{Flanigan2014}:
\begin{enumerate}
  \item 简单性:应该是一个简单图,即任意两个概念节点之间最多有一条边
  \item 连通性:应该是弱连通图
  \item 确定性:对于某个节点,不能有标签相同的两条出边
  \item 无环性:图中没有环(有向边形成的首尾相连的环)
\end{enumerate}

\section{本章小结}
本章主要介绍了文本理解的两种主要的研究类型,即篇章分析和语义理解。

篇章分析首先需要识别出大段文本中的基本篇章单位,可能是一个完整的句子,也可能是一个句子中的一个子句。子句通常可以由标点符号隔开,包括逗号、分号等等。其中,逗号的作用最为复杂,也是分割子句的标点中最常见的一种。本章介绍了在中文篇章分析中逗号功能的分类体系,将逗号分为前后句子成分有关和无关的两种;其中存在关系的一类逗号还分为并列关系、从属关系两种。在完成识别子句后,则需要根据句子中的连接词确定子句之间的篇章语义关系。

语义识别则包括浅层语义识别和深层语义识别两种。浅层语义识别包括语义角色标注任务,该任务需要识别出句子中的所有谓词,以及每一个谓词拥有的所有论元,一个谓词及其所有论元即是一个命题。而深层语义识别不仅识别出句子中谓词及其论元的关系,对于非谓词的词语与其他词语的关系也要进行识别。

本文所做的两项针对地理试题中选择题问句的理解工作,一项是基于选项中的逗号,对逗号隔开的成分之间的关系进行分类,将长句分割成可能的短句,简化后续句法分析及语义理解的输入;另一项是用AMR对选择题文本进行深层语义理解,试图找出其中有用的语义关系,帮助生成解题所需的语义模板。篇章分析和语义理解这两种文本理解研究是本文工作与通用文本理解工作之间的桥梁。

\chapter{地理选择题选项拆分}
\label{chapter:split}
\section{引言}
中文长句通常是一个复句,《现代汉语》对复句的定义为:复句是由两个或几个意义上紧密相关、结构上互不包含的分句构成的句子\cite{zhou2008}。这样的几个分句之间往往是由逗号分隔开的。根据上述定义,我们可以利用逗号在复句中的作用,将句子拆解成一些更简短的句子,从而降低句法分析、机器翻译、语义理解等等任务的难度。逗号除了可以分隔从句或是分句,在中文中也可以用来分隔并列的词语等等。

在地理试题中,尤其是选择题中,我们观察到一个现象:选择题的一个选项中(不考虑题面)经常包含一个或一个以上的逗号,而这个逗号常常隔开了两个或以上可以独立判断正误的陈述,但也有一些逗号隔开的是具有因果、递进等关系的子句篇章。如果我们可以判断出哪些逗号隔开的是相对独立的两件事情,哪些逗号不能用来作为分割点隔开长句中两个句法上并列的部分,则可以在合适的场景下,将较长的陈述句拆解成多个更简短的句子,从而减小例如句法分析及语义理解的分析难度。

在篇章关系分类中,这种可拆解的关系可以被视为是并列篇章关系。我们放宽了并列篇章关系的定义,不要求并列项在逻辑上完全不互相依赖,除非存在显式的连接词。在地理选择题中,选项中逗号隔开的两部分经常会有隐式的因果关系,因为后续的工作不要求对隐式因果关系做识别,实际上这种关系的识别也非常困难。在第\ref{geodatafeature}节中,会详细举例说明这种情况。

为了更好地处理选择题中选项包含了逗号的长句,我们根据逗号分开的两个句子部分的上下文特征,判断该逗号隔开的是否是句法上可以视作并列的两个部分,再用启发式的规则方法寻找公共部分右边界,进而将一些长句转换为几个短句。我们使用了最大熵模型进行分类,本章将介绍整个实验的方法及结果。

\section{中文句子分割的研究现状}
逗号是一种十分常见的标点符号,在中文文本中,逗号出现的频率比英文等语言更高,据统计每个英文句子中平均有0.869至1.04个逗号,而每个中文句子中平均有1.79个逗号\cite{Jin2004}。在中文的长句分析中,逗号可以起到十分重要的作用。在中文中,逗号不仅仅可以作为一个句子内部从句或者短语之间的停顿符号\cite{Li2004},也可以作为两个句法独立的句子之间的分割符\cite{Xu2013}\cite{Xue2011}。所以对于包含逗号的中文长句来说,利用逗号来将长句分解成更短的句子,可以对很多自然语言处理任务有比较好的提升作用,例如机器翻译\cite{Wang2014}、句法分析\cite{Jin2004}\cite{Li2004}\cite{MQ}\cite{Kong2014}\cite{Li2004}\cite{Li2008}等等。

Mei等人\cite{Jin2004}在文章中指出,中文的逗号中,约有30\%是用来将从句与主句或相邻的从句之间分隔开,逗号是一个中文句子的自然分割点,可以将逗号分割和句法分析结合起来。首先在合适的逗号位置将句子切分成几个短句,然后对每个短句分别做依存句法分析,再将短句之间用依存关系连接起来,得到原长句的句法分析结果。但不是所有的逗号都可以作为这样的分割点,有些逗号如果做为分割点,会导致一些词在短句内找不到head词,还有会导致一些词找到错误的head词。作者提出的方法认为,如果逗号分割的两个短句之间只存在一条依存关系边,则认为这个逗号是合理的分割点,通常这样的逗号出现在一个从句结束的地方。文章对每个逗号抽取了一些特征,使用SVM分类器对从句内逗号和从句间逗号进行分类,获得了87.1\%的准确率,实验结果显示可以使依存句法分析的性能提升9.6\%。

Xing Li等人\cite{Li2004}将标点符号看为分割标点和普通标点两种,前者可以将一个句子分割成几个子句。文章中主要使用了基于规则的方法来处理长句的句法分析。将句法分析分解成一个两步句法分析方法:首先用所有冒号、逗号、分号,将句子分成多个子句;然后对分割出的子句进行句法分析;再使用一种基于句法分析结果的规则的方法,判断出每个逗号是否分割一个并列结构,而不是多个从句,如果出现这种情况再使用规则的方法将这几个子句的句法树合并起来;最后在将每个子树的根节点的词性标签序列作为句法分析的输入,再做第二次parsing,结合前面的子树结果就可以得到原句完整的句法树。实验结果证明这种方法可以有效缩短长句句法分析的时间,并且可以将句法分析的性能提升7\%。

毛奇等人\cite{MQ}为了处理句法分析中的长句问题,提出了单独解析块的概念,即由特定的标点符号分割句子生成的自然词序列。单独解析块又分为可单独解析块和不可单独解析块,区别在于,前者的内部词序列在正确的句法树中只有一个根节点。文章思想类似Xing Li等人\cite{Li2004}的工作,但是由于使用规则的方法能够处理的情况比较局限,他们提出了一种基于统计的方法。因此将逗号分类的任务形式化为了对可单独解析块的识别问题。文中考虑了包括逗号在内的五种标点符号作为单独解析块的划分边界,提出一个特征集合,并使用了Id3决策树分类算法进行分类,对于可单独解析块的识别F值可以达到85.1\%,对不可单独解析块的识别F值为69.7\%。然后将单独解析块的识别加入句法分析的过程,首先对所有的可单独解析块进行句法分析,得到一个子树结构,然后在这些子树中抽取出其中的中心词与词性,再将这些中心词与其他不可单独解析块合并成一个词与词性序列,再对这个组合序列进行句法分析得到一棵全局句法树,最后再将之前得到的子树结构整合进最终的句法树中。实验结果表明,这种方法可以使句法分析效果在长度大于40的长句中,准确率提高1.59\%,召回率提高0.93\%,并且该方法有效缩短了句法分析花费的时间。

Jinhui Li等人\cite{Li2008}提出了另一种可以处理中英文句子的层次化句法分析的方法。文章从句法树森林中递归地识别出简单的组成成分,然后逐步减小句法树森林中句法树片段的个数,直至合并为一棵句法树。总体上大致分为三个步骤:词性标注、组块分析、句法分析。算法从由所有组块的句法树组成的森林开始,并且设计了一个BIESO标签体系,每一次句法树合并之前,都会使用最大熵模型从左到右地预测出每一棵子树的标签,然后对能够合并的连续子树进行合并,得到新的更小的句法树森林。这篇文章虽然并未直接涉及到利用逗号来作为句子的分割点,但是在组块分析中,实际上也用到了逗号的信息。

李艳翠\cite{liyancui2015}在研究中文篇章关系的工作中,从基本篇章单位的角度,对标点符号进行分类,分成篇章单位间标点和篇章单位内标点,其中篇章间标点中,逗号就占到了61.3\%。Yang Y.Q.等人\cite{Yang2012Chinese}将逗号的功能主要分为表明并列关系和从属关系的两类。其中表明并列关系的逗号包括三种:起到句子边界作用的逗号、分割父节点为非根节点的并列IP结构的逗号、分割并列动宾短语的逗号。另一种表明从属关系的分为三类:分割附属从句与主句的逗号、分割句子谓语与宾语的逗号、分割句子主语和谓语的逗号。这里更多地从语法功能的角度对逗号的作用进行了划分,而不是像上面的一些工作,完全从句法树的角度来对逗号进行分类。

总的来说,利用中文长句中的逗号进行句子切分,并没有一个十分明确的标准,有些研究是从句法树的角度出发,认为逗号是否能够切分句子,取决于人工标注的句法树是否将逗号前后两个部分表示成独立的子树;有些则是先在所有逗号处进行切分,再根据并列结构等的句法特点,识别出不应作为切分点的逗号并修正;还有些是从逗号在篇章切分中的作用出发,结合句法特点来对逗号进行分类。如何利用逗号来得到更简短的句子,要考虑到切分结果对某种应用场景(比如句法分析)的作用,也就是明确切分的目的是为了什么,并结合所处理的语料特点,选择合适的分类标准,从而用逗号将长句切分成短句。

\section{地理试题文本特点}
\label{geodatafeature}
选择题是高考地理试题中的一种重要的题型,占了大约一半的比重,在工作中,我们首先考虑对选择题的理解。本节先介绍一下地理试题的基本特点,以及我们怎样将选择题处理成完整的句子。

一道选择题通常由四个选项组成,每个选项可能是一句文字描述,也可能是一些小选项的组合(例如“\textcircled{1}\textcircled{2}\textcircled{3}”,小选项则属于题面的一部分)。由于我们重点是要理解选择题文本所表述的内容是否正确,所以通常是以“题面+选项文本”组成的完整句子作为一个待理解的对象。对于选项为小选项组合的情况,我们将小选项的内容找出来,和题面的其他部分拼接起来,得到一个完整的句子。

举个例子,一个不包含小选项的选择题如下:\par
(题面)该船即将进入\par
A 巴拿马运河   B 麦哲伦海峡\par
C 德雷克海峡   D 直布罗陀海峡\\


可以将题面与每一个选项拼接,得到四个完整的句子,然后判断这些句子的正误:\par
该船即将进入巴拿马运河\par
该船即将进入麦哲伦海峡\par
该船即将进入德雷克海峡\par
该船即将进入直布罗陀海峡\\

一个包含小选项的选择题如下:\par
(题面)岛内最大零售商业点位于甲村,主要形成原因是该村 \par
\textcircled{1} 地形平坦,交通便利   \textcircled{2} 商业从业人口多\par
\textcircled{3} 商业组织形式复杂     \textcircled{4} 人口数量大  \par
A \textcircled{1}\textcircled{2}  B \textcircled{1}\textcircled{4}\par
C \textcircled{2}\textcircled{3}  D \textcircled{3}\textcircled{4}\\

对于这类选择题,则不是将ABCD这些选项与原始句子直接拼接,而是将小选项内的文本与题面进行拼接,得到:\par
岛内最大零售商业点位于甲村,主要形成原因是该村地形平坦,交通便利\par
岛内最大零售商业点位于甲村,主要形成原因是该村商业从业人口多\par
岛内最大零售商业点位于甲村,主要形成原因是该村商业组织形式复杂\par
岛内最大零售商业点位于甲村,主要形成原因是该村人口数量大\\

我们提出的对地理选择题选项进行拆分的方法,就是将这样拼接出来的完整的一句话作为分类对象,在本文中我们称之为“试题文本”。拆分方法的提出主要基于选项中一个常见的现象,即选项经常是包含了由逗号隔开的几个短语或者动宾结构等等组成的部分,我们可以将这些部分拆分开来,分别与逗号两侧的IP或者VP等结构之前的公共部分拼接起来,使一个选项或者一个小选项对应的句子,拆分为多个更加简短的句子,这样得到的句子我们称为“拆分后试题文本”。举个直观的例子如下:

(题面)该海域沿岸 \par
A 存在上升流,为热带雨林气候   B 有暖流经过,为热带草原气候\par
C 有寒流经过,为热带沙漠气候   D 盛行东南风,为热带季风气候\\

这题原本可以组成四个句子:\par
该海域沿岸存在上升流,为热带雨林气候\par
该海域沿岸有暖流经过,为热带草原气候\par
该海域沿岸有寒流经过,为热带沙漠气候\par
该海域沿岸盛行东南风,为热带季风气候\\

但是可以直观感觉出,每一句话实际上都可以分解成两句来单独判断正误:\par
该海域沿岸存在上升流 \& 该海域沿岸为热带雨林气候\par
该海域沿岸有暖流经过 \& 该海域沿岸为热带草原气候\par
该海域沿岸有寒流经过 \& 该海域沿岸为热带沙漠气候\par
该海域沿岸盛行东南风 \& 该海域沿岸为热带季风气候\\

这样得到的每句话都更加简短,可以简化后续的句法分析、语义理解工作。在后续的地理试题理解中,我们的问答系统的项目为地理题制定了一套语义模板,需要在进行推理之前将自然语言表达的试题转换为模板表示。例如对于“该地区经济高速增长的根本原因是市场机制比较成熟”这个试题文本,对应的模板表示就是“原因(该地区经济高速增长,市场机制比较成熟)”。由于缺乏足够的标注数据,难以使用基于统计的机器学习方法来进行转换,为了提高转化的准确率和可信度,目前仍然是以基于规则的方法为主,这套规则考虑了词语、词性的特征,也考虑了句法分析结果中的特征。实验结果发现,这样的包含逗号的、含有句法并列成分的句子,是模板转换中错误率较高的一类。所以想办法将长的复杂句拆解成短的简单句,可以提高句法分析的性能,也可以提高基于句法结果的模板转换的准确率。

根据上述的动机,我们对切分的问题描述如下,以选项中只要一个逗号的情况为例:
\begin{enumerate}
	\item 如果我们可以将试题文本分成三个部分:公共部分+并列成分1+并列成分2,其中并列成分1和并列成分2之间的边界就是选项中的逗号,公共部分和并列成分的边界可能位于试题文本的选项逗号之前的任意位置,我们把这个边界叫做“公共部分右边界”。如“该海域沿岸@/存在上升流,为热带雨林气候”这个例子中,公共部分右边界就是“/”的位置,“@”表示题面选项边界,下文这两个符号的含义均与此相同。
	\item 然后根据公共部分右边界的位置重新组合得到两个简短的句子:\\
		-公共部分+并列成分1\\
		-公共部分+并列成分2\\
		并且这两个句子各自句法完整、语义通顺、两句合起来可以完整表达原试题文本的语义,则认为这样的试题文本是可拆分的。
\end{enumerate}

但显然,不是所有包含逗号的选项都可以像上面这个例子这样,根据逗号分成几个部分然后分别和题面拼接成一个完整的句子。我们的目标是将所有的试题文本分类成“可拆分”和“不可拆分”两类。

例如“符合图中该城区实际情况的表述是@北部地区的地租梯度,总体大于南部地区”,这个选项整体上是题面中“是”的宾语从句,而逗号前面的部分是这个从句中的主语,这个逗号是SBJ类型。我们不能拆分得到“符合图中该城区实际情况的表述是北部地区的地租梯度”和“符合图中该城区实际情况的表述总体大于南部地区”这两个句子,这样得到的句子不再通顺。再比如“当火炬传递到@甲地时,当地正值多雨季节”,逗号前面是时间状语,逗号类型为ADJ。

对于公共部分右边界的情况,根据定义也可以分成三种:与题面和选项的边界相同;位于题面内部;位于选项中逗号前的部分。但是位于题面内部这种情况在我们标注的数据中几乎没有出现过。所以这里给出一些试题文本的例子来说明另两种情况:
\begin{enumerate}
  \item 位于题面选项边界:根据左图中等温线分布特点可知,例如“该海区@/在北半球,A处有暖流经过”。
  \item 位于选项中逗号前:由08时到20时,例如“图中@甲地/风向偏北,风力逐渐减弱”。
\end{enumerate}

此外,在问题描述中,我们提及拆分后得到的句子应该能够表达原试题文本的语义,但是有一种情况例外,即试题文本的选项中,逗号隔开的两个部分之间,存在潜在的因果关系,即没有明确的关系连词,但从逻辑上来说存在一定因果相关性。例如“2000年到2012年,崇明县城镇化水平不断提高的主要原因是@/第一产业效率提高,农村出现剩余劳力”这句话中,实际上“第一产业效率提高”是“农村出现剩余劳力”的一个原因,这里有篇章上的隐式因果关系\cite{liyancui2015}。这里如果拆开成两个句子,则两者之间的隐式因果关系就不能体现出来。但我们仍然认为这样的试题文本是可拆分的,原因有二:一是前文中提到,做选择题拆分的目的是为了简化后续句法分析、语义理解的难度,尤其为试题模板换提供更加简短的输入,而在模板化中,原本就不会体现这种隐式的因果关系,而是将前后两个部分作为两个独立的事实来考虑;二是从自然语言处理的角度来说,这样的因果关系几乎不能从语法层面获知,只有基于一定的背景知识和具备一定的推理能力的前提下,才能发现这层隐式因果关系,所以识别难度很大,因此不做识别。

基于以上描述,我们结合在第\ref{commaclssify}中介绍的Yang Y.Q.等人\cite{Yang2012Chinese}对中文中逗号的功能分类,给出判断逗号所在试题文本是否可拆分的标准:
\begin{enumerate}
  \item SB:地理题选项中几乎没有此类逗号,不考虑。
  \item IP\_COORD:所在选项可拆分,逗号之前的IP的左边界为公共部分右边界,如“图中@甲地/风向偏北,风力逐渐减弱”。
  \item VP\_COORD:所在选项可拆分,逗号之前的VP的左边界为公共部分右边界,如“为防止艾比湖继续萎缩,在该湖流域应采取的措施是@/修建水库,调节径流”。
  \item ADJ/COMP/SUB:所在选项不可拆分,如一个SBJ类型的逗号的例子“符合图中该城区实际情况的表述是@北部地区的地租梯度,总体大于南部地区”。
\end{enumerate}

我们在实际的地理试题数据中,还发现了一些例外情况需要特别说明:
\begin{enumerate}
  \item 在Yang Y.Q.等人的分类中没有提及这样的逗号,即逗号前面是一个VP,后面是一个IP,可能因为省略了某些成分导致前面的VP不能形成IP,例如“英国@多数河流/短,含沙少”,这里“短”只能形成一个VP,而“含沙少”是一个IP,这个VP、逗号、IP三者的父节点都是根节点,实际上“短”应该是“长度短”。对于这种情况,我们也认为是可拆分的。
  \item 有时候选项中表述的两件事情是题面中明确提及的两个问题,题面中常包含关键字“分别是”“依次是”等等,这样的试题文本我们认为是不可拆分的,例如“与甲河段比较,乙河段的特点与主要成因分别是@流量大,雨水补给量大”。
\end{enumerate}

需要特别指出的是,ADJ/COMP/SUB等非并列关系的逗号,也可能出现IP\_COORD的句法形式,例如“玻利维亚@/受寒流影响,多雾少雨”这句话的句法结构如图\ref{parse_example}所示。对于“影响”类的句子,根据模板化处理的需要,认为是不可拆分的。

\begin {figure}
	\centering
	\includegraphics[width=0.8\linewidth]{{parse_example}.jpeg}
	\caption{COMP类型的逗号表现出IP\_COORD类型逗号的句法特点}
	\label{parse_example}
\end{figure}

所以,如何判断逗号的功能类型,不能仅仅从句法的角度判断,还是需要从语义本身出发,结合句法的特点来得到结论。综上我们给出了如何判断试题文本是否可拆分的直觉判断标准、从逗号功能角度出发的判断标准,基于上述论述,我们对地理试题库中的859道题进行了标注,在下一小节中将详细介绍标注数据的情况。

\section{选择题文本拆分}
\label{section:split_method}

我们将自动拆分分成两步来进行:首先判断含有逗号的选项是否可拆分;如果可拆分,再判断公共部分右边界的位置在哪里。由于标注数据中只有88个公共部分右边界不在题面选项边界的试题文本,所以使用统计机器学习的方法来预测这个边界不太可行,但通过观察到的数据特点,我们提出了一种简单的基于规则方法来解决这个问题。此外,语料中只有36个包含了一个以上逗号的选项,仅占所有含逗号选项的6.6\%,所以在实验中不考虑这类选项的分类,主要是针对仅包含一个逗号、两个选项部分的选项做分类,后面我们会使用一个简单的投票方法来解决项目实际使用中遇到的多逗号选项的分类问题。

下面我们会分几个小节分别介绍分类使用的上下文特征、后处理、基于规则的公共部分右边界识别、对于多逗号选项的分类。

\subsection{特征的选择}
\label{featuresection}
在对包含逗号的地理选择题选项的特点进行观察分析后,我们选择了如下18个特征作为最大熵模型\cite{MaxEnt1996}\cite{Ratnaparkhi1996A}的特征模板,并以“据/P 图/NN 推测/VV ,/PU 2003-2013年/NT 该/DT 市/NN@退耕还林/VV ,/PU 林地/NN 面积/NN 持续/VV 增加/VV”为例,说明每个特征在该例上的特征值:
\begin{enumerate}
  \item wordNumDiff:选项中逗号前后两部分词的个数的差值(例子中为3)
  \item charNumDiff:选项中逗号前后两部分字的个数的差值(例子中为4)
  \item postagEditDistance:选项中逗号前后两部分的词性序列的编辑距离(例子中为3)
  \item lastPosComb:项中逗号前后两部分最后一个词的词性组合(例子中为“VV/VV”)
  \item lastPosEqual:项中逗号前后两部分最后一个词性是否相等(例子中为True)
  \item firstPosComb:项中逗号前后两部分第一个词的词性组合(例子中为“VV/NN”)
  \item firstPosEqual:项中逗号前后两部分第一个词的词性是否相等(例子中为False)
  \item lastWordInTimian:题面的最后一个词(例子中为“市”)
  \item lastTwoWordsInTimian:题面最后两个词的组合(例子中为“该/市”)
  \item lastPostagInTimian:题面中最后一个词的词性(例子中为“NN”)
  \item timeCombination:项中逗号前后两部分是否包含时间词的布尔值组合(例子中为“False/False”)
  \item firstWordInSecondPart:项中逗号后部分的第一个词(例子中为“林地”)
  \item firstPostagInSecondPart:项中逗号后部分的第一个词的词性(例子中为“NN”)
  \item lastWordInFirstPart:逗号前部分的最后一个词(例子中为“退耕还林”)
  \item lastCharInFirstPart:逗号前部分的最后一个词的词性(例子中为“VV”)
  \item containCuewordsComb:项中逗号前后两部分是包含的线索词对应模板类型的组合(例子中为“None/None”)
  \item containCuewordsMain:选项中是包含的主要线索词(例子中为“None”)
  \item bothContainLonLat:项中逗号前后两部分是否包含经纬度的布尔值组合(例子中为“False/False”)
\end{enumerate}

其中,词性采用的是宾州树库的词性体系。此外,在特征中提到了“线索词”这个概念,线索词是指在试题语义模板转化过程中,使用的是基于规则的转换方法,根据一个线索词列表决定试题文本对应哪个语义模板,比如出现“利于”“导致”“使”等等词语,就会触发“影响”这个模板。这个线索词列表我们是从第\ref{chapter:tagger}章介绍的地理题标注系统中,标注的试题模板及线索词数据中抽取出来的。但有一些线索词对于是否可拆分的判断影响不大,甚至可能产生一些干扰,比如“时间限定”“运动”“构成”“指示”等模板的线索词。所以我们只关注一些比较重要的、能够区分是否可拆分的模板类型的线索词,包括“影响”“因素关联”“原因”这三类,包括如下这些词(或者词的组合):
\begin{enumerate}
	\item 因素关联:表示、因、导致、需要、与/有关、因此、不利于、促进/了、通过、若/可、便于、造成、应、因为、若/则、由于、越/越、降低
	\item 影响:控制、导致、拓展/了、受/影响、提高/了、提升/了、因素、借/优势、易/造成、宣传/了、利于、适宜、受/控制、受/制约、有利于、影响、便利/了、使
	\item 原因:得益于、产生/原因、原因/是、缘于、目的、主要、形成/原因、的/原因/是、成因、主要/是、因为、由于、的/目的/是、原因
\end{enumerate}

\subsection{分类结果的后处理纠正}
由于我们的训练数据比较有限,尽管有些我们考虑了某些文本特征并将其加入特征模板,有时候因为数据量有限,模型还是难以学出正确的分类。在我们观察到的结果中,大多数明显的可以用规则来解决的错误都是将不可拆分的误判成了可拆分的。所以我们针对一些明显的、具有规律性的错误分类样本,提出了一系列后处理纠错的规则,目前的规则均是用于将被预测为可拆分类的纠正为不可拆分类。包含下列9种情况:
\begin{enumerate}
	\item 选项中逗号后的部分的第一个词是某个特殊词,包括:利于,甚至,则,因此,便于,表示,但是,但,使,导致。这些词大多是线索词,或者是一些表示明确篇章关系的词语。例如:“符合图中该城区实际情况的表述是@东南方向地租等值线密集,表示该方向空气质量较好”。
	\item 选项中逗号前的部分的最后一个词是某个特殊词,包括:时。例如“下列对沿途地理现象的描述可信的是@经红海时,可见沿岸大片森林”。
	\item 选项被逗号隔开的两部分均包含表示经纬度的符号。例如“此时他可能位于@24°N,120°E”。
	\item 题面中包含某些特殊词,包括:分别,及。例如“图中河流的流向及河流与水渠的关系是@河流自南向北流,水渠水汇入河流”
	\item 选项被逗号隔开的两部分分别包含一组特殊的词中的一个,包括:越/越,若/则(其中第一个词出现在第一部分中,第二个词出现在第二部分中)。例如:“图中@大气中灰尘数量和颗粒越大,\textcircled{1}越多”。
	\item 选项中逗号前的部分包含某个特殊词,包括:因为,由于,因,借助。例如:“我国目前@因消费量少于生产量,原油可以大量出口”。
	\item 选项中逗号后的部分包含某个特殊词,包括:使、导致。例如:“大气中@臭氧层遭到破坏,会导致\textcircled{1}增加”。
	\item 选项中逗号前的部分包括某组特殊词,包括:受/影响,受/控制。例如:“玻利维亚@受寒流影响,多雾少雨”。
	\item 选项中逗号前的部分仅包含时间或者时间段,例如“仅考虑地球运动,图示窗户、屋檐的搭建对室内光热的影响有@春分到夏至,正午屋檐的遮阳作用逐渐增强”。
\end{enumerate}

对上述的后处理能够处理的场景,我们在特征模板中也尝试了将相关特征加入,例如逗号后部分的最后一个词、逗号前的部分的最后一个词、是否包含经纬度、线索词相关特征等等。但是可能由于我们的数据量比较有限,数据稀疏性比较大,所以学习出来的效果不太好。但也存在后处理过程将正确分类为本身为可拆分的样本错误纠正为不可拆分的样本,在实验结果分析中会详细说明。

\subsection{公共部分右边界的识别}
\label{boundary_rec}
标注数据中只有88个公共部分右边界与题面选项边界(下面简称“两个边界”)不一致的试题文本,占所有含逗号选项总数的22.5\%,没有足够的数据可以统计机器学习的方法来预测这个边界,因此我们通过对数据进行观察,总结了一些公共部分右边界位置的特点,提出了几点简单的基于规则的判定方法。

我们首先对公共部分右边界不在题面选项边界的88个试题文本进行了分析,发现在两个边界之间的词组类型分成下面几类:
\begin{enumerate}
	\item 地点:有52个样本,占59.1\%。我们将方向位置(“东部”、“地点”)、指代地点(“甲地”、“乙河”、“两地”)、描述性地点(“板块交界处”)、术语性地点(“平原地区”、“中游”、“河口处”、“城市中心”)均视为地点。例如:“巴西@亚马孙河/径流量大,流经经济发达地区”。
	\item 名词性术语:有24个样本,占27.3\%。例如:“东北平原@地势/中间高,南北低”、“在7、8月份,伦敦比北京@气温/高,日较差大”。
	\item 时间:有4个样本,占4.5\%。例如:“图l中,圣若阿金地区@终年/受赤道低压带控制,降水多”、“地震发生时的防震措施错误的是@在街上时/迅速离开电线杆和围墙,到开阔地躲避”。
	\item 地点+术语:有2个样本,占2.3\%。例如:“图中城市@上海市服务功能/强,辐射全国”,“上海市”为地点,“服务功能”为术语。
	\item 术语+时间,有1个样本,占1.1\%。例如:“下列关于该地逆温特征的描述,正确的是@逆温强度午夜/达到最大,后减弱”。
	\item 其他,有7个样本,占8.0\%。例如:“京津冀协同发展利于@城市间/的分工与协作,降低竞争力”、“此时,关于图中天气的正确叙述是@北京比东京/气温低,气压高”、“关于两国大豆产区及生产条件的叙述,正确的是@均/位于沿海地区,交通便利”、“读美国和巴西大豆生长周期表,下列叙述正确的是@都在/春、夏之交播种,秋季收获”。
\end{enumerate}

与上述提到的名词性术语相对的还有动词性术语,例如“退耕还林”“植树造林”等等,这种术语在所有术语中所占比例较小,并且在两个边界不一致的样本中没有出现过,所以我们暂不考虑。由上面的分析可以看出,如果两个边界不相同,那么两者之间的词组是比较有规律的。因此提出一个假设:如果选项中第一个词语是时间、地点、名词性术语或者其中两者的组合,那么公共部分右边界很可能位于它们的右边。于是我们继续对两个边界一致的样本进行观察,发现选项第一个词是上述时间、地点、名词性术语的同样不在少数:
\begin{enumerate}
	\item 名词性术语:例如“崇明县城镇化水平不断提高的主要原因是@/第一产业效率提高,农村出现剩余劳力。”、“甲地位于喜马拉雅山东端,林线高于青藏高原其它地区。其主要原因是@/纬度低,气温较高”。
	\item 地点:例如“关于‘丝绸之路经济带’东、西部沿海地区差异的描述,正确的是@/东部人口稠密,西部地广人稀。”
	\item 时间:例如“乙地气候特点是@/夏季高温多雨,冬季寒冷干燥”。
\end{enumerate}

我们综合上述结论发现,当选项的第一个词语是时间、地点、名词性术语(本段将这三类词语简称为“特殊词”)时:如果选项中逗号右边部分的第一个词语也是同类特殊词,那么很有可能两个边界是一致的(此时边界在逗号左侧部分的该特殊词的左边);如果选项中逗号右边部分的第一个词语不是对应类型的特殊词,那么公共部分右边界很可能位于选项开头的特殊词后面。对于两个边界之间的词组是特殊词组合的情况(如“术语+地点”),也可以在利用上述规则判断完选项开头的特殊词之后,再应用上述规则判断选项中第二个词语(例如“此次网购过程中@成都/正午太阳高度变小,白昼变短”,“成都”为地点,“正午太阳高度”和“白昼”为术语)。此外,当选项中第一个词语是动词时,则两个边界很有可能是一致的(在标注数据中,没有出现选项第一个词语是动词,而两个边界不同的情况);当选项中出现“均”“都在”时,公共部分右边界在这些词的后面。

这里我们只考虑选项中仅含一个逗号的试题文本。算法的输入为可拆分的试题文本的选项部分中,由逗号隔开的两个部分的分词及词性结果(对于时间/地点/术语这三类词,使用特殊的词性标记time/loc/term来表示)。我们将这个基于规则的识别方法用算法\ref{algorithm:boundary}来给出。

\begin{algorithm}
\begin{algorithmic}[1]
\STATE 输入:选项中逗号左边部分的词与词性$S_1=\{w_1, w_2, ..., w_m\}$,$P_1=\{p_1, p_2, ..., p_m\}$;选项中逗号右边部分的词与词性$S_2=\{w_{m+2}, ..., w_n\}$,$\P_2=\{p_{m+2}, ..., p_n\}$
\STATE 输出:边界右边第一个词的下标k
\STATE 初始化:k = 0
\FOR{$w_i$ in $S_1$}
	\IF{$w_i$ == “均” or “都”}
		\STATE k = i + 1
		\STATE break
	\ELSIF{$p_i$是动词词性}
		\STATE k = i
		\STATE break
	\ELSIF{$p_i$ in \{time, loc, term\} and $p_i$ != $p_{m+1+i}$}
		\STATE continue
	\ELSIF{$p_i$ in \{time, loc, term\} and $p_i$ == $p_{m+1+i}$}
		\STATE k = i
		\STATE break
	\ELSE
		\STATE k = i
    	\STATE break
    \ENDIF
    \STATE return k
\ENDFOR
\end{algorithmic}
\caption{\label{algorithm:boundary}公共部分右边界识别伪代码}
\end{algorithm}

\subsection{含多个逗号选项的处理}
在标注语料中,只有36个试题文本的选项包含了一个以上逗号,占所有含逗号选项的6.6\%。我们没有将这部分语料特殊处理后加入训练语料中。如前所述,我们所有的特征模板、后处理、公共部分右边界识别算法,都是针对选项中只含有一个逗号的试题文本的。对于选项包含一个逗号以上的情况,我们只在预测的时候做一些特殊处理,使得我们训练的模型和后处理等算法能够运用其上。

下文所述的算法只是解决试题文本自动拆分的第一步,即判断是否可拆分。对于公共部分右边界的识别,则直接扩展算法\ref{algorithm:boundary},在判断每个部分的第一个词是否属于同一类型特殊词时,将两个部分扩展为多个部分的第一个词之间的比较,只要有任意的另外一个部分与第一部分的开头有同类型的特殊词,就跳出循环。

对于含多个逗号选项的试题文本,判断其是否可拆分的算法思想类似于多分类问题在二分类模型下采用的one-to-one算法:首先用逗号将选项隔成多个部分$parts$=$\{part_1, ..., part_n\}$;然后将任意两个part用逗号组合在一起,得到一个仅包含一个逗号的伪选项,并与题面组合在一起得到伪试题文本;再使用训练得到的二分类模型(仅能处理选项含一个逗号的情况)对每种组合下的伪试题文本进行是否可拆分的二分类;最后统计对所有伪试题文本的判断情况,将占多数的判断结果作为原试题文本是否可拆分的判断结果,如果“可拆分”和“不可拆分”的伪试题文本数量一致,则倾向于不拆分,判断原试题文本为“不可拆分”。算法\ref{algorithm:multi_comma_algorithm}给出了上述过程的伪代码,其中的PREDICT()函数,就是使用最大熵算法得到的模型对一个选项仅含一个逗号的试题文本判断是否可拆分的函数。

\begin{algorithm}
\begin{algorithmic}[1]
\STATE 输入:题面T,选项被逗号隔开的多个部分$parts$=$\{part_1, ..., part_n\}$
\STATE 输出:原试题文本的是否可拆分
\STATE 初始化:判断为可拆分的伪试题文本数量$y\_num$=0;判断为不可拆分的伪试题文本数量$n\_num$=0
\FOR{i in \{1, ..., n-1\}}
	\FOR{j in \{2, ..., n\}}
		\STATE 伪试题文本 = 题面 + $part_i$ + "," + $part_j$
		\STATE predict\_result = PREDICT(伪试题文本)
		\IF{predict\_result == y}
			\STATE y\_num++
		\ELSE
			\STATE n\_num++
		\ENDIF
	\ENDFOR
\ENDFOR
\IF{$y\_num$ > $n\_num$}
	\STATE return 可拆分
\ELSE
	\STATE return 不可拆分
\ENDIF
\end{algorithmic}
\caption{\label{algorithm:multi_comma_algorithm}选项含多个逗号的试题文本是否可拆分判断算法}
\end{algorithm}

\section{实验及结果分析}
\subsection{实验数据}
本章实验使用的数据来自于第\ref{chapter:tagger}章中介绍的标注系统,共涉及55套试卷、859道选择题。由于高考问答系统的目标是解答北京高考地理试题,因此对标注试卷的选择尽量向北京高考试题靠近,避免不同试题风格的差异对系统性能带来干扰。这里的55套试卷包括:13套北京高考真题、11套其他地区高考真题、17套北京高考模拟题、14套北京期中/期末/会考/联考试题。

为了给试题文本拆分提供标注数据,我们用第\ref{chapter:tagger}章中介绍的标注系统,对55套试卷、共859道选择题进行了拆分的标注,包括是否可拆分的二分类标注,以及公共部分右边界的标注。除了拆分标注本身,我们还用到了这些试卷人工标注的词性、分词、时间的实体识别等数据。包括拆分标注在内,这些数据的标注将在\ref{chapter:tagger}中详细介绍。

\begin{table}[!htbp]
\begin{center}
\begin{tabular}{c|c}
\hline {数据类型} & {数量}\\
\hline 试卷 & 55\\
\hline 所有选择题 & 859 \\
\hline 所有试题文本(仅拆分前) & 3853\\
\hline 选项含逗号的试题文本 & 545\\
\hline 选项不含逗号的试题文本 & 3308 \\
\hline 选项仅包含一个逗号的试题文本 & 509\\
\hline 选项包含一个以上逗号的试题文本 & 36\\
\hline 可拆分的选项含逗号试题文本 & 391\\
\hline 不可拆分的选项含逗号试题文本 & 154\\
\hline 公共部分右边界即题面选项边界的可拆分试题文本 & 303 \\
\hline 公共部分右边界在选项中的可拆分试题文本 & 88 \\
\hline
\end{tabular}
\end{center}
\caption{\label{data_count} 拆分标注数据的统计信息}
\end{table}

表\ref{data_count}给出了标注数据的统计信息。可以看出,约有14.1\%(545/3853)的试题文本是含有逗号的,其中71.7\%(391/545)都是可以拆分的,在所有试题文本中占到10.1\%。所以从数据上来看,对地理试题做拆分,如果能够较好地利用拆分结果、尽可能多地找出可拆分的试题文本并拆分成短句,对句法分析和语义模板转换的性能将会有明显的影响。

但是我们的标注数据也存在两个问题:其一,尽管我们标注了859道题共3853个试题文本,但其中的85.9\%都是选项不包含逗号的,也就是说跟拆分无关,所以在训练分类算法的时候,我们可以使用的训练数据十分有限,仅有545句;其二,在这545句中,存在严重的数据倾斜问题,可拆分的数据占71.7\%而不可拆分的数据仅占到28.3\%,所以直接在这样的数据比例上进行训练,会导致不可拆分的数据的分类召回率较低,数据会倾向于被分类为可拆分的,但是错误地将不可拆分的句子预测为可拆分,会导致后面的句法分析和模板转换的输入没有意义,因此我们需要尽可能避免这种情况的发生。第一个问题是因为标注工作量较大(包括整理试题上传、标注分词和词性等),虽然可以使用自动的分词、词性、命名实体识别标注结果,但我们为了排除自动分析错误的影响,只使用了有人工标注词法数据的试题文本来做训练;第二个问题是由地理试题本身的语料特点决定的。

\subsection{是否可拆分二分类实验}
在是否可拆分的二分类实验中,我们使用python的nltk第三方工具包提供的最大熵模型工具作为分类器。在每一种配置下的实验中,我们都使用十折交叉验证来得到平均的性能;在每次实验中,训练集的比例占原始数据的90\%,测试集占10\%。

\subsubsection{特征的影响}
我们一共提出了18个上下文特征用于最大熵模型的训练和预测,在第\ref{featuresection}小节中做过详细介绍。本节我们不改变训练数据中两类数据的比例,直接将原始数据集中的所有选项含逗号的试题文本按9:1划分成训练集和测试集。

我们依次添加每一个特征,观察实验结果的变化情况。实验结果如表\ref{features}所示。可以看出,加了更多的特征之后总体的准确率反而出现了降低的趋势,同时我们比较关注的n\_recall指标出现了先上升后下降的趋势。实际上,n\_recall和total\_precision是一对此消彼长的特征,因为在测试数据中,可拆分类型的数据也远多于不可拆分类型的数据,如果对占多数的可拆分数据识别正确,则更可能会使总体准确率提升,此时不可拆分类型数据的召回率就更低。

\begin{center}
	\begin{table}[!htbp]
		\resizebox{\textwidth}{!}{
			\begin{tabular}{c|c|c|c|c|c}
				\hline {添加特征} & {y\_precision} & {n\_precision} & {y\_recall} & {n\_recall} & {total\_precision}  \\
				\hline +wordNumDiff & 89.6\% & 74.5\% & 90.3\% & 69.2\% & 84.8\% \\
				\hline +charNumDiff & 89.3\% & 71.8\% & 89.5\% & 68.5\% & 84.0\% \\
				\hline +postagEditDistance & 89.8\% & 73.2\% & 89.2\% & \textbf{70.0}\% & \textbf{84.2}\% \\
				\hline +lastPosComb & 90.4\% & 69.0\% & 86.8\% & \textbf{72.3}\% & 83.0\% \\
				\hline +lastPosEqual & 90.4\% & 69.0\% & 86.8\% & 72.3\% & 83.0\% \\
				\hline +firstPosComb & 90.0\% & 62.1\% & 82.0\% & \textbf{73.1}\% & 79.6\% \\
				\hline +firstPosEqual & 89.9\% & 62.1\% & 81.9\% & 73.1\% &79.6\% \\
				\hline +lastWordInTimian & 90.3\% & 61.2\% & 80.8\% & \textbf{74.2}\% & 79.2\% \\
				\hline +lastTwoWordsInTimian & 91.3\% & 56.7\% & 78.4\% & \textbf{78.5}\% & 78.4\% \\
				\hline +lastPostagInTimian & 91.3\% & 58.1\% & 79.5\% & 78.5\% & \textbf{79.2}\% \\
				\hline +timeCombination & 91.2\% & 59.1\% & 80.5\% & 77.7\% & \textbf{79.8}\% \\
				\hline +firstWordInSecondPart & 91.1\% & 61.4\% & 82.2\% & 76.9\% & \textbf{80.8}\% \\
				\hline +firstPostagInSecondPart & 91.0\% & 60.7\% & 81.6\% & 76.9\% & 80.4\% \\
				\hline +lastWordInFirstPart & 90.1\% & 61.3\% & 81.9\% & 76.2\% & 80.4\% \\
				\hline +lastCharInFirstPart & 91.0\% & 63.9\% & 83.8\% & 76.2\% & \textbf{81.8}\% \\
				\hline +containCuewordsComb & 90.8\% & 64.6\% & 84.1\% & 75.4\% & 81.8\% \\
				\hline +containCuewordsMain & 90.6\% & 65.3\% & 84.9\% & 74.6\% & \textbf{82.2}\% \\
				\hline +bothContainLonLat & 90.6\% & 65.3\% & 84.9\% & 74.6\% & 82.2\% \\
				\hline
		\end{tabular}}
		\caption{\label{features} 增加特征对性能的影响}
	\end{table}
\end{center}

这种趋势的出现,一方面可能因为训练数据量比较小,每种特征都能提高某一类试题文本的辨识度,但是这些类的文本数量还比较小,受稀疏性影响比较大,容易产生过拟合,对预测产生了一定的干扰;另一方面则可能是新特征在不可拆分数据上比较敏感,导致这类数据的召回率提高,相应地可拆分数据的准确率就变低了。

但不可拆分类的识别召回率比总体准确率更加重要,因此我们选出所有能使n\_recall升高的特征(wordNumDiff、postagEditDistance、lastPosComb、firstPosComb、lastWordInTimian、lastTwoWordsInTimian),或者没有使n\_recall发生变化但是不损失total\_precision的特征(lastPosEqual、firstPosEqual、lastPostagInTimian、lastCharInFirstPart),重新做一次实验,得到性能如表\ref{highnrecall}所示。


\begin{table}[!htbp]
\begin{center}
\begin{tabular}{c|c|c|c|c}
\hline {y\_precision} & {n\_precision} & {y\_recall} & {n\_recall} & {total\_precision}  \\
\hline 90.8\% & 58.1\% & 79.2\% & 76.9\% & 78.6\% \\
\hline
\end{tabular}
\end{center}
\caption{\label{highnrecall} 高n\_recall特征下的性能}
\end{table}

不过发现这里的n\_recall比表\ref{features}中的最好的n\_recall还要低。所以我们在后面的实验中还是选择表\ref{features}中最高的n\_recall对应的特征集来训练模型。

\subsubsection{训练集数据类型比例的影响}
在前文中我们曾提到过,对于试题文本的拆分我们倾向选择保守的拆分判断,即希望尽量能够提高不能拆分的试题文本的分类召回率。因为如果将不可拆分的试题文本进行拆分,会得到一些没有意义的句子,对后续任务比较不利;而将可拆分的试题文本判断为不可拆分,则只是保留了原句,下游任务的输入仍然是一个有意义句子。在实验数据中,选项含逗号的试题文本中有71.7\%(391/545)都是可以拆分的,存在比较明显的数据倾斜现象,模型也会倾向于将文本分类成可拆分的,这样就与我们希望提高不可拆分数据的识别召回率相矛盾。因此我们采用了重采样的方法,对训练数据中的不可拆分样本进行随机重采样,调整训练集中的两类数据的比例,测试集中数据比例保持不变。

在每次实验调整训练集的数据比例之前,还是先按9:1的比例划分训练集和测试集;如果我们想让训练集中的可拆分数据的比例超过71.7\%,则随机从中抽取一个可拆分的数据,复制该样本再加入训练集,直至比例达到所需。如果想让训练集中的可拆分数据的比例低于71.7\%,则重复从其中不可拆分的数据中随机抽取后复制放回。表\ref{yprop}中显示了不同的训练集可拆分数据比例y\_proportion下的各项性能。

该实验的出发点是在高考地理题理解的实际应用中需要n\_recall尽可能高,即尽量不要将不可拆分的数据误识别为可拆分,同时total\_precision也越高越好。但前文已经说过这两个指标是负相关的。从表\ref{yprop}中的结果来看,我们选择可拆分数据在训练集中的比例为40\%应用在实际使用中,这个比例下,n\_recall接近90\%,比原始比例提高了10.7\%,total\_precision比原始比例降低了6.2\%。

\begin{center}
	\begin{table}[!htbp]
		\resizebox{\textwidth}{!}{
			\begin{tabular}{c|c|c|c|c|c}
				\hline {y\_proportion} & {y\_precision} & {n\_precision} & {y\_recall} & {n\_recall} & {total\_precision}  \\
				\hline 10\% & 96.1\% & 38.3\% & 44.6\% & 94.6\% & 57.6\% \\
				\hline 20\% & 96.0\% & 42.0\% & 53.2\% & 93.1\% & 63.6\% \\
				\hline 30\% & 95.7\% & 46.7\% & 62.2\% & 91.5\% & 69.8\% \\
				\hline 40\% & 95.2\% & 50.1\% & 67.3\% & 89.2\% & 73.0\% \\
				\hline 50\% & 93.5\% & 51.7\% & 70.8\% & 85.4\% & 74.6\% \\
				\hline 60\% & 92.4\% & 52.5\% & 73.0\% & 82.3\% & 75.4\% \\
				\hline 70\% & 91.3\% & 53.7\% & 75.4\% & 79.2\% & 76.4\% \\
				\hline 原始比例 & 91.3\% & 58.1\% & 79.5\% & 78.5\% & 79.2\% \\
				\hline 80\% & 90.2\% & 59.0\% & 81.4\% & 74.6\% & 79.6\% \\
				\hline 90\% & 89.2\% & 69.5\% & 88.1\% & 69.2\% & 83.2\% \\
				\hline 
		\end{tabular}}
		\caption{\label{yprop} 不同的训练集中可拆分数据比例下的性能}
	\end{table}
\end{center}

\subsubsection{后处理的影响}
基于前两个实验,本实验对可拆分数据在训练集中的比例为原始比例和40\%两种情况,给出使用后处理和不使用后处理的性能比较,如表\ref{postprocess}所示,“正确纠正”指将应该是不可拆分的数据由原本预测为可拆分纠正为不可拆分的次数,“错误纠正”指将应该是可拆分的数据由原本预测为可拆分纠正为不可拆分的次数。

实验结果表明后处理过程对n\_recall有明显提升作用。在训练集40\%的可拆分数据比例下,后处理过程能使n\_recall提升28.5\%,总体识别准确率提升4.4\%。不过需要说明的是,由于我们的不可拆分数据较少,后处理规则是根据数据的特点提出来的,所以可能存在过拟合的现象,在更多的未知数据中的提升不一定有这么明显。

\begin{center}
\begin{table}[!htbp]
\resizebox{\textwidth}{!}{
\begin{tabular}{c|c|c|c|c|c|c|c|c}
\hline {可拆分数据比例} & {实验类型} & {y\_precision} & {n\_precision} & {y\_recall} & {n\_recall} & {total\_precision} & {正确纠正} & {错误纠正}  \\
\hline 40\% & 无后处理 & 85.4\% & 44.2\% & 71.6\% & 62.3\% & 69.2\% & - & - \\
\hline 40\% & 有后处理 & 95.8\% & 50.7\% & 67.6\% & 90.8\% & 73.6\% & 32 & 17\\
\hline 原始比例 & 无后处理 & 79.9\% & 47.8\% & 86.2\% & 37.7\% & 73.6\% & - & -\\
\hline 原始比例 & 有后处理 & 92.2\% & 58.8\% & 79.5\% & 80.8\% & 79.8\% & 56 & 25\\
\hline
\end{tabular}}
\caption{\label{postprocess} 后处理的影响}
\end{table}
\end{center}

但该过程也同时会将很多可拆分的试题文本错误纠正为不可拆分的,我们观察了具体的结果后发现主要有两种情况:(1)由标注错误引起的错误,例如“图中河流的流向及河流与水渠的关系是@河流自南向北流,河流水补给水渠”被标注为可拆分,实际应该是不可拆分的,因为题面问的是两个问题,如果只和后面的一个部分拼接成句子,则题面中有一部分内容是无用的;(2)由于相似的句式在不同句子在标注时出现了偏差引起的错误,例如“由08时到20时,图中@\textcircled{2}地受高压脊控制,天气持续晴朗”被标注为可拆分,“大红门、动物园服装批发市场@受历史因素影响,形成于传统商业区”、“孟加拉国的降水特征对农业生产的影响是@旱季持续时间长,利于作物成熟”被标注为不可拆分。从我们对拆分的定义上,这些句子应该分类为可拆分(得到两个短句各自有意义且不改变原句意),但从直觉上来说又有一点模糊,前后两部分之间的因果关系较明显,所以对这类错误的纠正是可以接受的。实际上,在考虑模板化工作的需求,这些句子均应标注为不可拆分,所以也可以看成是标注错误。

所以,这两种后处理中出现的主要错误都是可以接受的,不认为会对使用性能产生明显负面影响。

\subsection{公共部分右边界识别实验}
\begin{table}[!htbp]
	\begin{center}
		\begin{tabular}{c|c|c|c}
			\hline {边界类型} & {正确的个数} & {总数} & {正确率}  \\
			\hline 与题面选项边界相同(A类) & 267 & 303 & 88.1\% \\
			\hline 在选项第一部分中(B类) & 71 & 88 & 80.7\% \\
			\hline 所有数据	& 338 & 391 & 86.4\% \\
			\hline
		\end{tabular}
	\end{center}
	\caption{\label{boundary} 公共部分右边界识别算法性能}
\end{table}
本节使用第\ref{boundary_rec}节中介绍的基于规则的方法,对数据集中所有可以拆分的数据进行公共部分右边界的识别。实验结果如表\ref{boundary}所示。实验结果表明该方法是可行的。对识别错误的样本给出如下分析:
\begin{enumerate}
	\item 第二部分开头的术语是第一部分开头术语的一个属性,将公共部分右边界错误识别为题面选项边界:例如“图示两区域@河流/以雨水补给为主,流量稳定”,“流量”是“河流”的属性,而不是与其地位相同的另一个术语。这类错误共有6例。
	\item 多种边界均可接受:例如“1980至2000年该城市人口密度的变化表现在@甲区/人口密度最高,增长速度最快”,我们的算法会将边界识别在“人口密度”后面,但是因为题面中出现过这个词,所以认为第二部分可以不用补全这个概念,两种边界都是可以接受的。这类错误共有16例。
	\item 第二部分中有省略的隐含术语:例如“由08时到20时,图中@\textcircled{2}地/气压升高,未来呈降温趋势”,第二部分隐含了术语“气温”。再例如“巴西@亚马孙河/径流量大,流经经济发达地区”,第二部分没有一个明确可表述的隐含术语,但从语义上来说是描述了亚马逊河的另一个和“径流量”同等地位的属性。这类错误共有12例。
	\item 第二部分中包含了时间词,干扰了同等地位术语的判断:例如“下列关于图中等温线与甲、乙、丙、丁四地气温的叙述,正确的是@甲地/海拔高,1月气温低于丙地”,“海拔”和“气温”是等地位的,但是第二部分中的“1月”干扰了算法的判断。这类错误共有2例。
	\item 类似术语但不是术语的词出现在开头:例如“企业总部留在北京主要因为@/距离远,搬迁费用高”中的“搬迁费用”。这类错误共有2例。
	\item 其他:例如“京津冀协同发展利于@城市间/的分工与协作,降低竞争力”中,边界会被算法识别为“城市”之后、“间”之前,但这样得到的第二个句子就不再通顺。这类错误不具有典型性,出现较少,其他类别不包括的错误都归于此类,这类错误共有15例。
\end{enumerate}

\section{本章小结}
本章针对高考地理问答系统中的选择题的特点,提出了将复杂的试题文本(一个题面+选项组成的完整句子,选项中含有逗号)拆分成多个简单句的方法。逗号在中文中使用广泛,经常用来对短语或者从句进行并列,如果选项是一个包含逗号的较为复杂的句子,则如果能判断选项是否说的是两件或多件可以独立判断真伪的事情,然后将每个部分和题面分别组合成更简短的句子,这样对后续的语义模板转换的任务则有比较大的帮助。根据项目组相关人员的实验结果,在语义模板转换的错误分析中,由于选项较为复杂(含逗号,陈述多件事)导致转换失败或错误的比重较大。因此如果我们能够尽可能识别出不能拆分的句子,同时识别出一些可以拆分的句子并将其简化,则对语义模板的性能改进有较大意义。

在本章中,我们介绍了地理选择题的特点,并详细阐述了试题文本拆分的直观定义,以及与中文逗号分类的关系,并介绍了我们标注的所有拆分相关的数据的情况。然后将拆分过程分成两个主要步骤:识别是否可拆分、识别公共部分右边界。在第一步中,我们使用最大熵模型,利用试题文本的上下文特征来训练分类模型,并提出了提高不可拆分试题文本识别召回率的方法:一是提高训练集中不可拆分试题文本的比例,来抵消数据倾斜导致召回率低的问题;二是使用后处理过程,通过基于规则的方法,将一些样本的识别结果纠正为不可拆分。在第二步中,我们通过对地理试题数据的观察,提出了基于规则的方法来寻找公共部分右边界。

在第一步的分类中,在训练使用40\%的不可拆分数据比例,使用使召回率最高的10个特征,并使用9个后处理规则,使得不可拆分数据的召回率可以达到90.8\%,可拆分数据的召回率达到50.7\%,也就是说对不可拆分数据的识别错误很少的情况下,能找出一半的可拆分数据,这对后续的语义模板转换仍然是十分有意义的。而在第二部公共部分右边界的识别中,总体上可以达到了86.4\%的准确率。

目前的性能仍然有提升的空间,例如获取更多的数据,对本章提到的错误分析进行一些针对性的处理等等。

\chapter{AMR语义理解}
\label{chapter:amr}
\section{引言}
在AMR出现以前,自然语言语义理解的工作还比较零散,例如实体连接、命名实体识别、语义角色标注等等工作,从某种意义上来说都属于语义理解的研究范畴。每一项研究都只关注了语义理解的一个方面,而没有一个公认的完整的语义表达体系,能够将句子中的语义完整地描述出来。AMR则希望能够解决这个问题,它是一种图表示方式,能将句子中的语义从表层的语法表示中抽象出来,将动作、实体、修饰等等各种语法要素都抽象成概念,然后以边来表示出概念之间的关系。

AMR作为一种新型的语义表示方法,目前看来是一种比较强大的、有潜力的语义表示,可以给自然语言理解带来一种新的研究方向,并对一些相关领域的研究或者工程性的工作带来结果上的提升\cite{kai2015improving}\cite{Pan2015}。因此,本文针对高考地理问答系统中的地理试题的理解,考虑了AMR这种新方法,但由于AMR在中文的处理上暂时还是空白,可使用的工具和语料还很有限,目前还未进入实际应用阶段。我们有幸和中文AMR语料标注小组合作,得到了更多可用的中文语料。本文针对AMR的主要工作是对比现有算法在不同语言(中英文)、不同语料类型、人工对齐与否等方面的性能,并在少量的地理试题AMR标注数据上进行了一些实验,展示AMR在中文上应用的现状和存在的问题,为将来在试题理解任务上的应用打下基础。

\section{AMR的研究现状}
句法树库对自然语言处理领域的发展具有巨大的影响力,比如宾州树库就是一个典型的例子。但是在语义标注方面,目前已有的标注语料还比较分散,比如有单独的命名实体、指代消解、语义关系、篇章关系等等,目前还缺少一个能够将整个句子的语义逻辑关系组合在一起的语义标注树库。AMR就是在这样的背景下被提出的,这是一种能够将句子语义表示成一个单根的简单有向图的的表示方法。在这种有向图中,节点表示一个概念,通常是句子中的一个词语或者词组,或者在词语或词组的基础上抽象出来的概念,有向边表示节点之间的关系,边具有指示概念间语义关系的标签。这种表示体系的提出,以及基于这个体系的语料库的建立,可能会给自然语言理解的任务带来新的发展空间。

AMR表示是一种图表示方法\cite{Banarescu2013Abstract},对人来说,AMR标注是易读的,同时对于程序来说,也很容易获取到该表示中的所有信息。AMR的目标是能够从句子的不同的句法表达方式中,抽象出句子的语义,也就是说对于不同表达方式的语义相同的句子,希望能够得到相同的AMR表示结果。在AMR表示中,用到了大量PropBank框架的内容,对具有多个表示框架的谓词,会在概念节点中注明对应的是该谓词的哪种用法,在标注它的论元时,也会在边上标记出相应的论元序号。

L Banarescu等人\cite{banarescu2012abstract}在2012年提出了第一个版本的AMR标注规范,明确了AMR应该如何标注,并给出了大量的标注示例,说明了怎么选择根节点、节点的内容应该怎样确定、关系标签的类型、常见句式中如何添加新的抽象节点及标注关系标签、如何标注命名实体和数字时间及其它各类型实体等等。在这个标注规范的基础上,L Banarescu等人\cite{Banarescu2013Abstract}在2013年公布了一个英文的AMR标注图库,包含大约5000句标注文本。

在中文方面,李斌等\cite{Li2016Annotating}在《小王子》语料上标注了一个AMR图库,总共包含1562个句子,并且一个5000句以上的基于CTB的AMR图库正在标注中,本文也用到了这个新的语料。中文的语法表达比英文更加随意,例如有时会出现一个谓词在句子中不是连续的字序列,而是被别的词语分割开来(例如“帮了很大的忙”中的谓词“帮忙”)。他们对一些中文AMR标注中特殊之处进行了详细说明,成为AMR应用于中文的一个开创性的工作。此外,在英文的AMR标注中,没有标注图中的概念节点与原句中的词语的对齐关系,通常借助自动对齐算法来进行对齐,因此可能会损失一定的准确性。在这项中文AMR标注工作中,还加入了对齐信息的标注,包括概念节点的对齐以及少部分边的对齐信息。

有了AMR语料库之后,陆续出现了一些AMR自动解析的算法。最早的公开工作是来自Flanigan等人\cite{Flanigan2014}在2014年提出的一种两阶段的图算法,将AMR解析分成概念节点识别和概念节点间的边预测两个步骤。对于边的预测,采用了一种类似最大生成树算法的方式,得到所有概念节点间相互连接形成的连通图。Angeli等人\cite{Angeli2014}对于AMR子图的生成提出了一种健壮性更强的方法。随后在2015年,Wang Chuan\cite{Wang2015}等人发现AMR的表示方法有一部分与句法分析的结果比较相似,受到基于转换的句法分析算法的启发,他们提出了一种基于转换的AMR解析算法,并设计了一套适用于AMR图生成的转换方法:在句子的依存句法分析结果的基础上,每次预测一种转换动作,一步步将句法分析的结果转换成AMR的表示。Pust等人\cite{Pust2015}于同年提出了使用基于语法的机器翻译的方法来做AMR的解析,这篇研究将英文句子到AMR表示的转换看成是一种string到tree的翻译过程,并设计了一种AMR表示下的语言模型。Lucy Vanderwende等人\cite{Vanderwende2015}的工作则支持对英文、法文、德文、西班牙文、日文的AMR解析,他们设计了一些逻辑形式(Logical Form)到AMR的转换规则,通过这样的方法来实现对上述语言的解析,这些语言还没有可用的AMR语料库,因此这篇文章对于AMR在其他语言中的扩展也有重要的借鉴意义。

AMR作为一种语义表示方法,被寄希望于提高多项nlp任务的性能。目前已发表的研究中,有将AMR用于提高事件检测任务的性能\cite{kai2015improving},Xiang Li等人利用AMR自动分析算法,对待检测事件的文本进行AMR解析,然后将AMR结果中的一些信息作为特征加入,实验结果证明这样的做法可以在原来的基础上提高2.1\%的F值。还有研究将AMR用于无监督的实体链接任务\cite{Pan2015},结果表明使用了AMR信息的无监督实体链接的性能可以和有监督的实体链接性能相当。

目前已知的AMR英文语料规模达到了4万多句,中文的大规模语料库预计也将于不久之后公开。随着更多可用语料的出现,对AMR自动解析和将AMR应用于其他自然语言处理任务的研究会越来越多,因此AMR是一种具有潜力的语义表示方法。尽管目前在中英文上的AMR自动解析效果还不尽如人意,但是AMR的出现为语义分析和自然语言理解提供了一个全新的研究方向。

\section{AMR背景知识}
\label{amrback}
\subsection{标注体系}
AMR主要有两个部分组成:概念(节点)、概念间的关系(边)。下面将从这两个方面介绍AMR的标注体系。

\subsubsection{概念}
AMR图中的“概念”主要有三种来源:句中词语(例如图\ref{amrgraph}中的“boy”)、propbank中的语义框架(例如图\ref{amrgraph}中的“want-01”)、抽象的关键字。

从之前的例子可以看出,AMR试图对不同的表达方式甚至是相同语义的不同词语做出统一的标注方式。因此英文AMR对常见的concept制定了一个实体类型的标准列表,首先需要从这些实体名称中找出和想要描述的实体最相近的一个(例如有人会说“person”,有人会说“woman”,则可以通过这个列表统一起来):
\begin{enumerate}
  \item person, family, animal, language, nationality, ethnic-group, regional-group, religious-group
  \item organization, company, government-organization, military, criminal-organization, political-party, school, university, research institute, team, league
  \item location, city, city-district, county, local-region, state, province, country, country-region, world-region, continent, ocean, sea, lake, river, gulf, bay, strait, canal, peninsula, mountain, volcano, valley, canyon, island, desert, forest, moon, planet, star, constellation
  \item facility, airport, station, port, tunnel, bridge, road, railway-line, canal, building, theater, museum, palace, hotel, worship-place, market, sports-facility, park, zoo, amusement-park
  \item event, incident, natural-disaster, earthquake, war, conference, game, festival
  \item product, vehicle, ship, aircraft, aircraft-type, spaceship, car-make, work-of-art, picture, music, show, broadcast-program
  \item publication, book, newspaper, magazine, journal
  \item natural-object
  \item law, treaty, award, food-dish, disease
\end{enumerate}

有时候句子中也会有单词指示该实体的类型,如果这个词比上述方式选择的类型更具体,则使用这个词代替,例如对于“the poet William Shakespeare”,可以用:
\begin{lstlisting}[language=C]
   (p / poet
       :name (n / name :op1 "William" :op2 "Shakespeare"))
\end{lstlisting}

如果上述列表中所有实体名都没有合适的,句子中也没有明确指明实体的类型名称,就可以用“thing”来表示这个实体。例如对于“Words are the source of misunderstandings”,AMR中对“understanding”这个添加了“thing”这个concept:
\begin{lstlisting}[language=C]
   (s / source-01
       :ARG1 (t / thing
           :ARG0-of (m / misunderstand-01))
       :ARG2 (w / word))
\end{lstlisting}

如果一个实体有多个类型,例如“著名诗人、画家XXX”,则可以使用:mod关系来修饰,例如对于“the poet Dr. Seuss”:
\begin{lstlisting}[language=C]
   (d / doctor
       :name (n / name
           :op1 "Seuss")
       :mod (p / poet))
\end{lstlisting}

还有一类概念用来表示数量:monetary-quantity , distance-quantity , area-quantity , volume-quantity , temporal-quantity , frequency-quantity , speed-quantity , acceleration-quantity , mass-quantity , force-quantity , pressure-quantity , energy-quantity , power-quantity , voltage-quantity (zap!-, charge-quantity , potential-quantity , resistance-quantity , inductance-quantity , magnetic-field-quantity , magnetic-flux-quantity , radiation-quantity , concentration-quantity , temperature-quantity ,score-quantity ,fuel-consumption-quantity , seismic-quantity。通常这类概念后面都会有一个quant关系指向表示具体数值的节点。

此外还有一些常见的concept列举如下:
\begin{enumerate}
	\item relative-position(相对位置)
	\item product-of和sum-of(数学运算)
	\item date-entity(时间)
	\item date-interval(时间区间)
	\item percentage-entity(百分比)
	\item phone-number-entity(电话号码)
	\item email-address-entity(电子邮箱)
	\item url-entity(url)
\end{enumerate}

对于特殊疑问句,AMR还对被提问的部分设计了一个amr-unknown概念。例如“What did the girl find?”:
\begin{lstlisting}[language=C]
   (f / find-01
       :arg0 (g / girl)
       :arg1 (a / amr-unknown))
\end{lstlisting}

\subsubsection{关系}
AMR在英文标注中大约有100种关系,比较常见的可以大致分成以下5个类型:
\begin{enumerate}
	\item 框架论元 :arg0,:arg1,:arg2,:arg3,:arg4,:arg5.
	\item 通用语义关系 :accompanier, :age, :bene ciary, :cause, :compared-to, :concession, :condition, :consist-of, :degree, :destination, :direction, :domain, :duration,:employed-by, :example, :extent, :frequency, :instrument, :li, :location, :manner, :medium, :mod, :mode, :name, :part, :path, :polarity, :poss, :purpose, :source, :subevent, :subset, :time, :topic, :value.
	\item 数量相关的关系 :quant, :unit, :scale
	\item 时间实体相关的关系 :day, :month, :year, :weekday, :time, :timezone, :quarter, :dayperiod, :season, :year2, :decade, :century, :calendar, :era.
	\item 列举关系 :op1, :op2, :op3, :op4, :op5, :op6, :op7, :op8, :op9,
:op10.
\end{enumerate}

对于所有的关系,AMR还允许这些关系的反转形式也作为关系标签,例如:arg0的反转形式为:arg0-of,:location的反转形式为:location-of。
除了上面列举的,还有一些其他的关系标签,不再一一列举。

\subsection{语料对齐}
\label{aligndata}
AMR的语料对齐,即是将原句中的词语、词组与AMR图中的一个概念节点或者一个概念节点子图(例如一个时间概念子图,包括一个date-entity概念及其指向的表示年、月、日等信息的节点)进行对齐,对齐结果对于自动解析算法有着重要的作用,例如在概念识别等阶段。图\ref{amr_align}给出了一个AMR节点与句子词语之间的对齐示意图。

\begin {figure}[!htbp]
	\centering
	\includegraphics[width=1.0\linewidth]{{align}.jpeg}
	\caption{AMR对齐}
	\label{amr_align}
\end{figure}

在语料建设工作中,不同的语料可能存在一个差异,即是否有对齐信息。目前英文的语料都是不含有对齐信息的,上文的例子都是这一类标注;中文已公布的1500句小王子语料也是不含有对齐信息的,但是即将公开的一些中文数据将对齐信息考虑了进去,嵌入了AMR的标注中。本文的AMR部分工作是与含有对齐信息的语料标注团队合作,所以在我们的实验中也使用到了这一部分数据。

下面给出同一个中文句子的有对齐标注和无对齐标注两个版本,来说明对齐信息是如何标注的。这个例子是对来自《小王子》语料中的一句话的两种标注,这句话分词的结果是“画 的 是 一 条 蟒蛇 正在 吞食 一 只 大 野兽 。”:

\begin{lstlisting}[language=C]
无对齐标注:
   (x12 / 吞食-01
       :arg0  (x14 / 蟒蛇
           :quant  (x15 / 1)
           :cunit  (x5 / 条))
       :arg1  (x16 / 野兽
           :mod  (x17 / 大-01)
           :quant  (x18 / 1)
           :cunit  (x10 / 只))
       :domain  (x19 / thing
           :arg1-of  (x20 / 画-01))
       :time  (x21 / 正在))
\end{lstlisting}

\begin{lstlisting}[language=C]
有对齐标注:
   (x8 / 吞食-01
       :arg0  (x6 / 蟒蛇
           :quant  (x4 / 1)
           :cunit  (x5 / 条))
       :arg1  (x12 / 野兽
           :mod  (x11 / 大)
           :quant  (x9 / 1)
           :cunit  (x10 / 只))
       :domain  (x24 / thing
           :arg1-of  (x1 / 画-01))
       :aspect  (x7 / 正在))
\end{lstlisting}

可以看到,在AMR的整体结构上两者基本一致,差异在于变量(variable)名的选取。

对于无对齐信息的标注版本,在英文语料中,变量名通常是概念的首字母,如果出现了重复的首字母,则在后面添加一个数字来区分,例如“b2”等等。在中文语料中则通常是一个字母“x”加上一个数字,这个数字没有特别含义,只要能够将不同的概念区分开来即可。

对于有对齐信息的标注版本,每个变量名中的数字都是有含义的。如果概念对应原文的某个词或者短语,那么变量名的数字后缀就是对应词语的下标;如果概念是新增的抽象概念,例如对于时间实体加了“date-entity”概念节点或者上面增加的“thing”这个概念,他们的数字下标是任意一个超过句子长度的数字,表示该概念不直接对应于句子中的词语,以免混淆。

在中文的对齐标注中,根据概念与句子分词结果中的对齐方式分类,有这几种情况:
\begin{enumerate}
	\item 对应一个词语:例如上面的“(x11 / 大)”,表示“大”这个概念对齐到句子的第11个词
	\item 对应多个词语:例如“ 那么 刺 有 什么 用 呢 ? ”的AMR标注中有一个概念“x4\_x6 / 有用-01”,表示“有用-01”这个概念对齐到句子的第4和第6个词上。对于连续的词,也需要逐个写出对应的区间内的每一个词的下标,以免和上面这种跳词现象混淆。
	\item 对应某个词语中的某几个字:例如“我 把 锤子 、 螺钉 、 饥渴 、 死亡 , 全都 抛在 脑后 。”的AMR标注中有一个概念“x12\_1 / 抛-01”,表示“抛-01”这个概念对齐到句子的第12个词的第一个字上(如果是多个字则依次写出每个字在词中的下标,例如“x5\_1\_2\_3”)。
\end{enumerate}

关于自动对齐,Flanigan等人\cite{Flanigan2014}在提出JAMR解析算法的同时,为了训练概念节点的识别,提出了一种自动对齐的算法,主要是根据一系列的规则,用贪心算法来进行对齐,达到了92\%的准确率、89\%的召回率和90\%的F值,但是这种对齐方式只支持对概念节点和词语之间进行对齐,而没有考虑到边的label可能会来源于某个词语,也存在对齐关系。Nima Pourdamghani等人\cite{Pourdamghani2014}提出了另一种自动对齐方式,该方法主要分为预处理、训练、后处理三步,在预处理阶段将AMR转换成一个字符串(保留关系的标签),然后进行小写化、去除停用词、去除词缀、去除等预处理,然后利用IBM的对齐模型进行训练,在后处理再重建出对齐好的AMR图。在对齐过程中,这种方式包含了边的标签信息,所以可以对边也进行对齐。

\subsection{自动分析算法}
\label{jamr}
Flanigan等人\cite{Flanigan2014}在2014年提出了一种两阶段的图算法JAMR,这是第一个公开的AMR自动分析算法,为后面的研究提供了一个较强的baseline。在2015年Wang Chuan\cite{Wang2015}等人观察到AMR表示与句法分析结果之间的相似性,提出了一种基于转换的AMR分析算法。Pust等人\cite{Pust2015}也在同年提出了一种使用基于语法的机器翻译的方法进行AMR的解析。Lucy Vanderwende等人\cite{Vanderwende2015}还提出了一种基于逻辑形式(Logical Form)到AMR的转换规则的分析方法,来处理没有大规模AMR标注语料的语言的AMR分析问题。本节重点介绍本文实验基于的JAMR图算法。

Flanigan等人\cite{Flanigan2014}将AMR分析分成两个阶段:第一个阶段是从句子中识别出概念节点;第二个阶段是预测这些节点之间的边,也就是概念之间的关系。

在概念识别阶段,句子将会被切分成多个连续的片段(span),可通过序列标注算法来实现,并将每一个片段对应于一个概念子图(AMR的一个片段,可能包含多个概念节点,例如一个时间实体包括一个date-entity概念和具体的年月日概念等等),这些概念子图都来自于训练语料中这些片段曾经对齐到的概念子图,或者是一个空图(即放弃这个片段到AMR的对应)。通过对一系列连续的句子子串\textbf{b}和一系列的概念图片段\textbf{c}进行打分(两者的个数均为k),使用公式\ref{concept_iden}进行优化,其中\textbf{$f$}是句子子串(span)和它可以对应的一个概念子图片段在上下文中的特征向量表示,包括词、长度、命名实体等等这些特征,k为句子被切分成的片段数。

\begin{equation}
	\label{concept_iden}
	score(\textbf{b}, \textbf{c}; \textbf{$\theta$})=\sum_{i=1}^{k}\textbf{$\theta$}^\mathrm{T}f(\textbf{w}_{b_i-1\:b_i}, b_{i-1}, b_i, c_i)
\end{equation}

在关系识别阶段,基于第一个阶段识别出来的概念子图片段,在子图之间添加关系得到最终的AMR结果。这个结果需要满足五个条件,其中四个是在本章开头介绍过的AMR的有向图需要满足的四个条件:连通性、简单性、确定性、无环性,另一个就是保持性,即在第一阶段识别出所有概念子图片段应该是最终的AMR结果的子图。该阶段中,首先会对训练语料中的边训练一个线性参数的函数,特征为这条边的一些上下文,在解码的时候根据训练出的参数和函数对候选边进行打分。算法描述如下:边的候选集合为所有概念节点之间两两连接的有向边(两个方向,所有可能的边类型);首先对于第一阶段识别出来的概念子图中已经存在的边,相应的节点对之间的其它边都不再作为候选边,这些存在的边成为AMR最终结果的组成部分;再根据训练出的边打分函数,对未确定关系的节点对之间的所有候选边打分,保留分数最高的一条边,其他的不再作为候选;将所有得分大于0的边选出作为AMR最终结果的组成部分;对剩余的候选边,根据得分从高到底依次判断每条边是否进入最终结果,如果这条边可以将之前未相连的两个子图连接起来,则保留这条边;直至所有的节点之间形成一个弱连通图。

\subsection{自动评价}
目前,AMR最常用的评价方式是计算自动生成的AMR和测试集中的参考AMR的smatch得分\cite{Flanigan2014}\cite{Cai2013Smatch}。

\begin {figure}
	\centering
	\includegraphics[width=\linewidth]{{two_amr}.jpeg}
	\caption{两个待比较的AMR结果}
	\label{twoamr}
\end{figure}

在本章开头曾经介绍过,AMR可以以逻辑三元组集合的形式表示出来,例如对于图\ref{twoamr}中左边的AMR,可以表示成:
\begin{lstlisting}[language=C]
   instance(a, want-01) ^ 
   instance(b, boy) ^ 
   instance(c, go-01) ^ 
   ARG0(a, b) ^ 
   ARG1(a, c) ^ 
   ARG0(c, b)
\end{lstlisting}

三元组有两种形式,一种是relation(variable, concept),另一种是relation(variable1, variable2),例如上述前三行是第一种三元组,后三行是第二种三元组。

对于图\ref{twoamr}中右边的AMR,可以表示成:
\begin{lstlisting}[language=C]
   instance(x, want-01) \^
   instance(y, boy) \^ 
   instance(z, football) \^
   ARG0(x, y)\^
   ARG1(x, z)
\end{lstlisting}

smatch就是计算两个AMR表示的三元组之间匹配的准确率、召回率和F值(通过将两个AMR中的变量名映射起来,然后计算两组三元组之间的相同的个数,再根据总的三元组个数计算准确率和召回率)。困难在于两个AMR所使用的变量名并不相同,所以两个AMR之间基于不同的变量映射方式,可以计算得到多种重合结果。smatch则定义为:将两个AMR之间的节点一一对应起来,所能得到的最大的F值。例如针对上面的例子,有六种变量匹配的方式,如表\ref{smatch_table}。

\begin{table}[!htbp]
\begin{center}
\begin{tabular}{c|c|c|c|c}
\hline {映射方式} & {M} & {P} & {R} & {F} \\
\hline x=a, y=b, z=c & 4 & 4/5 & 4/6 & 0.73 \\
\hline x=a, y=c, z=b & 1 & 1/5 & 1/6 & 0.18 \\
\hline x=b, y=a, z=c & 0 & 0/5 & 0/6 & 0.00 \\
\hline x=b, y=c, z=a & 0 & 0/5 & 0/6 & 0.00 \\
\hline x=c, y=a, z=b & 0 & 0/5 & 0/6 & 0.00 \\
\hline x=c, y=b, z=a & 2 & 2/5 & 2/6 & 0.36 \\
\hline smatch score & \multicolumn{4}{r}{0.73}\\
\hline
\end{tabular}
\end{center}
\caption{\label{smatch_table} 不同变量映射方式}
\end{table}

所以这个例子的smatch得分就是所有映射方式中所能得到的最大的F值0.73。虽然smatch的概念很简单明了,但是计算起来实际上没有这么简单,对于包含大量节点的AMR,枚举所有的可能的映射方式的时间复杂度很高,所以有一些高效的算法被用于计算近似的smatch值,例如爬山法等等,这超过了本文的介绍范围,所以不再展开。

\section{中文AMR分析}
据我们所知,目前还没有针对AMR中文语料进行自动解析的算法及实验结果。在本节中我们将介绍如何中文语料建设的现状,如何将AMR自动解析算法应用到中文数据上。我们的分析算法部分是基于Flanigan等人\cite{Flanigan2014}提出的两阶段的图算法JAMR,对中文处理部分替换一些模块,以及对人工对齐语料进行对齐信息的提取,以满足JAMR的需求。

\subsection{中文语料建设}
目前,在中文上已发表的AMR语料,已知的只有1500句左右的小王子无对齐标注语料\cite{Li2016Annotating};有幸与李斌老师带领的中文AMR标注小组合作,我们获得了有对齐标注的小王子语料、有对齐标注的基于CTB语料的AMR标注。

在地理试题上,我们也获得了一个729句地理选择题试题文本(题面+一个选项组成的完整句子)的AMR标注数据。地里试题是一类语义明确的文本,因此也比较适合标注语义表示。一个例子如下:
\begin{lstlisting}[language=C]
# ::id test_amr.50 ::2017-03-22 22:01:22
# ::snt 中山站 到 南极点 的 直线 距离 约 为 2070 千米
(x13 / distance-quantity
      :quant  (x9 / 2070)
      :unit  (x10 / 千米)
      :domain  (x6 / 距离
            :mod  (x5 / 直线)
            :mod  (x2 / 到-01
                  :arg0  (x1 / 中山站)
                  :arg1  (x3 / 南极点)))
      :mod  (x7 / 约))
\end{lstlisting}

\subsection{对齐标注的转换}
\label{subsection:dataalign}
在第\ref{aligndata}小节,我们详细介绍了对齐的含义,以及有对齐的标注语料是如何标注这个信息的。对齐信息将会用于AMR解析的第一步,即concept识别阶段;此外,实际上这个对齐信息就是JAMR算法第一阶段想要得到的结果,也可以直接将这个对齐结果作为第二阶段预测的输入,以此来单独观察算法在两个阶段的性能表现。

英文语料中没有标注对齐,JAMR中使用了一种基于规则的自动对齐方法来对齐,可以达到约90\%的F值\cite{Flanigan2014}。例如对于句子“Why should any one be frightened by a hat ? ”的AMR标注为:
\begin{lstlisting}[language=C]
   (f / frighten-01
       :ARG0 (h / hat)
       :ARG1 (o / one
           :mod (a / any))
       :ARG1-of (c / cause-01
           :ARG0 (a2 / amr-unknown)))
\end{lstlisting}

JAMR会得到下面这样的对齐信息:\par
\# ::alignments 5-6|0 3-4|0.1 2-3|0.1.0 8-9|0.0 \par
\# ::node	0.0	hat	8-9  \par
\# ::node	0.1	one	3-4  \par
\# ::node	0.1.0	any	2-3  \par
\# ::node	0.2	cause-01	  \par
\# ::node	0.2.0	amr-unknown  \par	
\# ::root	0	frighten-01 \par
\# ::edge	cause-01	ARG0	amr-unknown	0.2	0.2.0	\par
\# ::edge	frighten-01	ARG0	hat	0	0.0	 \par
\# ::edge	frighten-01	ARG1	one	0	0.1	 \par
\# ::edge	frighten-01	ARG1-of	cause-01	0	0.2	 \par
\# ::edge	one	mod	any	0.1	0.1.0	 \\

对齐信息分成三类:(1)以“\# ::node”开头的几行:节点编号、该节点的concept、该concept在原句中对应的子串的下标范围(包括位置,不包括终止位置,下标从0开始,例如“5-6”表示句子下标为5的词,即“frightened”。这个范围称为“span”);(2)以“\# ::edge”开头的几行:边的出发节点的concept、边的指向节点的concept、边的出发节点的编号、边的指向节点的编号;(3)以“\# ::alignments”开头的一行:所有在“node”行出现的span及其对应的节点的编号。

有时,一个span可以对应多个节点,例如对某个概念抽象出另一个概念,如“(q / question-01)”这个概念上抽象了一层得到“(t2 / thing:ARG1-of (q / question-01))”,此时“thing”和“question-01”这两个概念节点对应到同一个span上。这种情况下,用“+”来表示对应多个节点,如“14-15|0.3.1+0.3.1.0”。

还有更特别的情况,例如“The flowers have been growing thorns for millions of years .”中的“years”,可以对应到AMR标注中的两个节点,即“t2 / temporal-quantity”和“y / year”,而“temporal-quantity”概念除了“year”这个子节点,还有别的子节点。该例的AMR标注为:
\begin{lstlisting}[language=C]
   (g / grow-03
       :ARG0 (f / flower)
       :ARG1 (t / thorn)
       :duration (m / multiple
           :op1 (t2 / temporal-quantity
               :quant 1000000
               :unit (y / year))))
\end{lstlisting}
                        
对于一个span对应多个节点的情况,人工对齐的数据中只会标注出原词直接对应的概念与原词的对应信息(如“year”这样的节点),而对于抽象出的节点如“temporal-quantity”、“thing”等等,则没有对齐标注。例如原句有10个token,这样的抽象概念的变量名可能是“x15”,这样就不能对齐到原句的词语上,也就是无对齐标注。在我们从人工标注的对齐语料中转换对齐信息格式的时候,对这种情况也只对齐出有对齐标注的节点,对于抽象的节点则不产生对齐结果。这一点可能会对JAMR后续的性能产生负面的影响。

对于有对齐标注的语料,我们将其转换成JAMR需要的形式,但是在转换过程中我们发现中文对齐与英文有所不同,中文的对齐有一些更复杂的场景:

\begin{enumerate}
	\item 一个概念对齐到不连续的几个词语上,例如“有 什么 用”中可以提取出概念“有用-01”。
	\item 一个概念对齐到一个词的一部分,例如“市场分析家”中的“分析”可以作为一个概念。
	\item 概念间的关系有时候也与原句的一些词语存在对齐关系,例如“他 大胆 地 向 国王 提出 了 一 个 请求”中,“大胆”会被对齐到一个概念上,该概念与其父节点的关系是“:manner”,而“地”字表明了“大胆”表示一种方式,所以“地”可以被对齐到“:manner”这个关系上。
\end{enumerate}

这里提出这三个问题,是未来工作中的可改进之处。由于目前中文CTB的AMR标注数据还不够完善,尤其是对于前两种对齐情况还有较多的标注错误,所以目前还无法提出有意义的方法来解决这些问题。由于JAMR现有模型还不能处理这些对齐情况,我们对这几种情况在工程实现上作了一些兼容的处理:对于第一种情况,我们将概念视为对齐到多个词语中的第一个词语上;对第二种情况,视为对齐到原词;对第三种情况,忽略所有边对齐信息。这样的让步会丧失这些对齐信息的作用,但好在这样的对齐在语料中所占的比重很小,所以对算法的性能不会产生太明显的影响。

如果对中文数据直接使用JAMR的自动对齐算法,在大部分情况下也可以成功对齐。主要的问题是,JAMR的概念对齐对于时间使用了只适用于英文的一套正则表达式,并不能直接在中文中使用。但是在语料中,时间所占的比例还是非常小的,所以暂时可以忽略这个原因给性能带来的影响。在实验中,我们也会将这个自动对齐算法应用于中文上。

\subsection{将JAMR应用于中文}
我们使用JAMR的算法来做中文的AMR分析,其算法思想在第\ref{jamr}小节中有详细介绍。

用JAMR训练以及分析新句子时,首先需要进行预处理,一是调用一些自然语言处理基础任务的工具包对语料进行分析,包括得到语料的分词(英文不需要分词)、命名实体识别结果、依存句法分析结果等,二是使用基于规则的算法来对齐AMR标注图中的concept和原句中的字符串,用于概念识别。我们没有改动JAMR的主体算法,只对这两种预处理针对中文做了一些调整,以使后续的工作可以在中文数据上奏效。在下面两个小节中,我们详细介绍这些工作内容。

上文中已经提到,JAMR的预处理中需要得到分词、命名实体识别、依存句法分析这三项基本自然语言处理任务的结果,以用于后续的训练和预测。

对于分词,英文不需要进行分词,而中文的分词结果在标注语料中已经给出,所以我们只需关注另外两项处理。

对于命名实体识别,JAMR原本使用的是IllinoisNER,该工具能够识别出PER(人名)、LOC(地名)、ORG(机构名)三种命名实体;在中文处理中,我们使用哈工大LTP的命名实体识别,对应地,该工具可以识别Nh(人名)、Ns(地名)、Ni(机构名)。

对于句法分析,JAMR原本针对英文的处理和我们对中文的处理,都是使用了StanfordParser,但是我们需要使用中文的模型,我们使用的是xinhuaFactored.ser.gz这个模型。

\section{实验及结果分析}
在本节中,我们会给出原始JAMR在英文上的结果,以及修改后的JAMR在中文上的结果。还会给出在中文上对使用JAMR自动对齐方法和人工标注对齐数据的效果对比。并且会在中英文语料上做封闭测试,以验证该算法模型的有效性。

\subsection{实验数据简介}
对于英文测试,我们会使用Flanigan等人\cite{Flanigan2014}在发表JAMR的论文中使用的数据、小王子语料;对于中文测试,我们会使用小王子语料、CTB语料、地理数据语料。具体数据集情况如表\ref{amrdatas}所示。其中,LDC2013E117是一个新闻语料,CTB中文语料则是来源于论坛、博客。

\begin{table}[!htbp]
\begin{center}
\begin{tabular}{c|c|c|c|c}
\hline {数据集} & {总句数} & {训练集} & {开发集} & {测试集} \\
\hline LDC2013E117(英文) & 8252 & 3988 & 2132 & 2132 \\
\hline 小王子(英文) & 1562 & 1274 & 145 & 143 \\
\hline 小王子(中文) & 1562 & 1274 & 145 & 143 \\
\hline CTB(中文) & 5003 & 4097 & 414 & 492 \\
\hline 地理选择题(中文) & 729 & 583 & 73 & 73 \\
\hline
\end{tabular}
\end{center}
\caption{\label{amrdatas} AMR中英文语料说明}
\end{table}

\subsection{封闭测试}
我们对表\ref{amrdatas}提到的五个中英文数据集都做了封闭测试,将表格中说明的测试集和训练集的数据合并起来,作为封闭测试的训练和测试数据,开发集保持不变。实验结果如表\ref{closetest}所示。该测试对于中文,是使用JAMR自带的自动对齐算法做概念的对齐。

这里我们主要对比中英文小王子语料的实验结果,因为这两个语料是平行语料,可以排除句子长度、句子类型(口语化还是正式)、语料规模带来的影响。从实验结果来看,对于英文的AMR解析,仅使用小王子1500句的语料,JAMR的两阶段测试总体F值可以达到73.0\%,可以说明该模型对于AMR的结构具有一定的处理能力;如果直接使用标注AMR对齐出的概念作为概念识别的结果,只进行第二阶段的预测,F值可以达到84.4\%。而对中文AMR的解析,两阶段测试的F值只有49.2\%,比英文低23.8\%;仅预测第二阶段的F值为54.0\%,比英文低29.6\%。

smatch(gold)概念指标可以反映第二阶段预测的情况,Span识别则反映第一阶段识别的情况。从实验结果可以看出,中英文的第一阶段识别性能差不多,而第二阶段(关系)识别则拉开了很大的差距,第二阶段的识别与句子内部的结构关系紧密,与句法特征等等比较有关。

\begin{center}
	\begin{table}[!htbp]
		\resizebox{\textwidth}{!}{
			\begin{tabular}{c|c|c|c|c|c|c|c|c|c}
				\hline {数据集} & \multicolumn{3}{c|}{Smatch(两阶段)}  & \multicolumn{3}{c|}{Smatch(gold概念)}  & \multicolumn{3}{c}{Span识别} \\
				\hline  & P & R & F-score & P & R & F-score & P & R & F-score \\
				\hline LDC2013E117(英文)& 81.5\% & 82.2\% & 81.8\% & 95.9\% & 87.9\% & 91.7\% & 82.1\% & 90.2\% & 85.9\% \\ 
				\hline 小王子(英文)& 74.9\% & 71.2\% & 73.0\% & 92.3\% & 76.0\% & 83.4\% & 79.5\% & 94.1\% & 86.2\% \\
				\hline 小王子(中文)& 56.3\% & 43.8\% & 49.2\% & 66.4\% & 45.6\% & 54.0\% & 83.5\% & 92.4\% & 87.8\% \\
				\hline CTB(中文) & 57.8\% & 39.7\% & 47.1\% & 68.4\% & 49.8\% & 57.6\% & 85.4\% & 80.2\% & 82.7\% \\
				\hline 地理选择题(中文) & 70.1\% & 58.7\% & 63.9\% & 74.3\% & 62.6\% & 67.9\% & 94.5\% & 94.0\% & 94.3\% \\
				\hline
		\end{tabular}}
		\caption{\label{closetest} AMR中英文封闭测试性能}
	\end{table}
\end{center}

在Levy等人的工作\cite{Levy2003Is}中,论述了对中英文进行句法分析的难度差异。在Durrett等人在2015年的工作\cite{Durrett2015Neural}中,对比了多种句法分析方法在英文句法分析上的性能,其中最高的F值可以达到92.4\%,是使用集成方法达到的。而在Zhang等人2011年的工作\cite{Zhang2011Syntactic}给出的多个方法的性能比较中,在CTB语料上进行句法分析的最好F值为80.4\%。中文句法分析在性能上与英文仍有明显的落后,而句法特征是AMR训练和预测使用的重要特征,因此句法分析上的错误是中文AMR解析性能明显差于英文的一个重要原因。

此外,可能JAMR用来识别英文概念之间的关系时,使用的一套特征模板可能不能很好适用于中文,需要重新针对中文数据调试出一套性能更好的模板。中文是一种意合的语言,而英文是形合的语言,中文的语义表达可能更加多变。

这里仅对中文性能较差的原因给出了一些推测,我们的实验结果仅仅是在中文上进行AMR解析的一次初步尝试,性能差异的具体解释和探求已经超出了本文的讨论范围。

\subsection{开放测试}
\begin{center}
\begin{table}[!htbp]
\resizebox{\textwidth}{!}{
\begin{tabular}{c|c|c|c|c|c|c|c|c|c}
\hline {数据集} & \multicolumn{3}{c|}{Smatch(两阶段)}  & \multicolumn{3}{c|}{Smatch(gold概念)}  & \multicolumn{3}{c}{Span识别} \\
\hline  & P & R & F-score & P & R & F-score & P & R & F-score \\
\hline LDC2013E117(英文) & 67.6\% & 60.8\% & 64.1\% & 84.6\% & 77.6\% & 81.0\% & 75.0\% & 74.3\% & 74.7\% \\
\hline 小王子(英文) & 57.6\% & 39.8\% & 47.1\% & 75.1\% & 59.0\% & 66.1\% & 74.2\% & 66.9\% & 70.4\% \\
\hline 小王子(中文) & 44.6\% & 31.3\% & 36.8\% & 58.9\% & 42.4\% & 49.3\% & 72.7\% & 71.0\% & 71.9\% \\
\hline CTB(中文) & 42.7\% & 31.2\% & 36.1\% & 57.5\% & 41.0\% & 47.8\% & 74.1\% & 68.5\% & 71.2\% \\
\hline 地理选择题(中文) & 51.9\% & 24.6\% & 33.3\% & 63.5\% & 51.4\% & 56.8\% & 84.4\% & 50.5\% & 63.2\% \\
\hline
\end{tabular}}
\caption{\label{generaltest} AMR中英文开放测试性能}
\end{table}
\end{center}

\begin{center}
	\begin{table}[!htbp]
		\resizebox{\textwidth}{!}{
			\begin{tabular}{c|c|c|c|c|c}
				\hline 句子长度范围	& len<10 & 10<=len<20 & 20<=len<30 & 30<=len<40 & len>=40 \\
				\hline CTB(中文) & 10.0\% & 37.8\% & 31.6\% & 12.5\% & 7.2\% \\
				\hline 小王子(中文) & 42.3\% & 42.1\% & 10.9\% & 2.8\% & 2.0\% \\
				\hline 地理选择题(中文) & 32.0\% & 57.6\% & 9.9\% & 0.5\% & 0.0\% \\
				\hline
		\end{tabular}}
		\caption{\label{sentencelength} 中文语料句长分布统计}
	\end{table}
\end{center}

该测试对于中文,也是使用JAMR自带的自动对齐算法去做概念的对齐。值得注意的是,三个中文语料的两阶段测试F值都在30\%至40\%之间,而他们的语料规模有明显差异。如表\ref{amrdatas}所示,CTB语料共有5003句,小王子语料有1562句,地理题语料有729句。我们猜测这样的结果是由于三个语料的句子长度差异导致的,句长统计结果如表\ref{sentencelength}所示。

CTB语料的平均句长为21.8,小王子的平均句长为12.9,地理试题的平均句长为12.4。统计结果为之前的猜测提供了一些证据。我们可以初步得到一个结论,平均句长较大的语料,AMR的解析会更难做。

\subsection{自动对齐与人工对齐}
对齐信息是新旧AMR标注之间最主要的区别,因此我们想验证一下目前这种对齐标注的有效性。在本实验中我们对比了三个中文数据集使用自动对齐和人工对齐的实验结果,如表\ref{align_exp}所示。

从实验结果来看,使用人工对齐不一定能够提高实验结果,例如小王子语料的Span识别,和地理选择题的Span识别以及gold概念下的smatch得分,都有不同程度的下降。性能下降可能有这几方面因素:一是自动对齐本身就有90\%以上的准确率,二是人工对齐忽略了对新增抽象概念的对齐。另外还有一些在第\ref{subsection:dataalign}小节中提到的一些中文对齐的问题。

\begin{center}
\begin{table}[!htbp]
\resizebox{\textwidth}{!}{
\begin{tabular}{c|c|c|c|c|c|c|c|c|c}
\hline {数据集} & \multicolumn{3}{c|}{Smatch(两阶段)}  & \multicolumn{3}{c|}{Smatch(gold概念)}  & \multicolumn{3}{c}{Span识别} \\
\hline  & P & R & F-score & P & R & F-score & P & R & F-score \\
\hline 小王子(自动对齐) & 44.6\% & 31.3\% & 36.8\% & 58.9\% & 42.4\% & 49.3\% & 72.7\% & 71.0\% & 71.9\% \\
\hline 小王子(人工对齐) & 40.7\% & 35.3\% & \textbf{37.8}\% & 59.0\% & 50.7\% & \textbf{54.5}\% & 63.8\% & 63.2\% & 63.5\% \\
\hline CTB(自动对齐) & 42.7\% & 31.2\% & 36.1\% & 57.5\% & 41.0\% & 47.8\% & 74.1\% & 68.5\% & 71.2\% \\
\hline CTB(人工对齐) & 44.9\% & 34.3\% & \textbf{38.9}\% & 57.1\% & 45.9\% & \textbf{50.9}\% & 75.7\% & 71.2\% & \textbf{73.4}\% \\
\hline 地理选择题(自动对齐) & 51.9\% & 24.6\% & 33.3\% & 63.5\% & 51.4\% & 56.8\% & 84.4\% & 50.5\% & 63.2\% \\
\hline 地理选择题(人工对齐) & 48.9\% & 26.6\% & \textbf{34.4}\% & 62.1\% & 50.1\% & 55.5\% & 72.5\% & 46.2\% & 56.4\% \\
\hline
\end{tabular}}
\caption{\label{align_exp} 中文AMR对齐方式的影响}
\end{table}
\end{center}

\section{本章小结}
AMR是一种新型的语义表示方法,目前在英文上开展了一些研究,包括语料标注、自动对齐、自动分析等等,但是在中文上仅有一个1500句左右的小王子语料是公开的,另外还有一些CTB语料、本文研究背景相关的地理选择题语料正在标注中。本章工作基于英文AMR的第一个公开的自动解析工具JAMR,将其中一些对英文处理的组件替换成中文,使其可以处理中文语料。

本章工作主要介绍了JAMR针对中文的改动、对中文有对齐标注的处理,并对中英文共5个语料进行了实验,实验包括封闭测试、非封闭测试、对齐数据对比测试。实验结果发现,JAMR对英文AMR的描述能力比较强,封闭测试可以达到81.8\%的总体性能,而对中文AMR的封闭测试只有不到50\%,可能是由于中文本身句法分析的准确性就比英文有明显区别,错误累积使得中文的AMR效果差得更多。在开放测试中,中文的实验结果平均比英文也要低20\%左右。在中英文小王子的实验结果对比中发现,主要是在第二阶段关系识别时,中文数据的表现远差于英文,而概念识别的性能则相差不多。另外通过中文三个语料实验结果的对比,发现句子长度越长的语料,即使语料规模更大,也会比较难以得到更好的结果。在语料对齐方面,我们比较了在中文数据上使用JAMR的自动对齐和直接使用标注好的对齐数据两种方式的实验结果,发现人工标注的对齐不总是能提高实验结果。

本章工作主要是对AMR在中文上的应用做一次探索,尝试现有的算法在中文数据上的应用,并测试了在高考地理选择题数据上的应用效果。可以看出高考地理题仅用较小的语料规模就可以得到与5000句以上CTB语料差不多的实验效果。实际上,700多句高考选择题文本中有很多是非常类似的,因为一道选择题有四个选项,在很多情况下,各个选项之间可能仅有几个词语发生了变化。所以实际上地理题的标注规模是很小的。因此可以猜测,AMR在地理题这类句子较短、语义清晰、书面化的语料上,有比较好的应用前景,是一个值得探索的应用领域。

在未来工作中,除了可以尝试不同的AMR分析算法在数据上的预测效果,还可以进一步利用有对齐信息的标注数据,结合对齐标注和自动对齐,将抽象出的概念也进行对齐,使对齐标注的作用不会被丢失的抽象概念对齐抵消掉。在JAMR的算法中,对概念的识别比较依赖训练集中的出现过的概念,几乎不能识别未在语料中出现的概念,可以针对这一点对概念的识别做一些优化,例如对实词,即使未在训练语料中出现过,也将其转换为一个概念节点。在中文AMR分析中,关系识别的效果远差于英文的现象,也应该引起注意,思考一些针对中文的改进方法,比如寻找适用于中文的特征模板等。

\chapter{地理试题标注系统}
\label{chapter:tagger}
\section{引言}
为了给选择题试题文本拆分和地理试题的AMR分析提供标注语料,也为了给整个高考问答系统的各项涉及自然语言处理和理解的任务提供高质量的地理试题标注语料,我们开发了一个地理试题标注系统。除了标注选择题选项拆分和AMR的数据,这个系统还支持对分词、词性标注、命名实体、术语、成分句法分析等传统自然语言处理任务在地理试题上的标注,以及试题语义模板、套话、上下文、题干核心成分分类、选项类别、题干前导部分分类等地理试题领域相关的、用于解题的各项数据的标注。

在此之前,我们没有统一的数据标注平台和规范,导致得到的数据比较零散、错误率高,我们开发这个系统主要是为了提供下列各项服务:
\begin{enumerate}
	\item 由于各个任务所需要的数据之间互有关联(如词性标注依赖分词的结果),但通常不同的标注内容由不同的人员完成,例如分词是一个同学负责,语义模板是另一些同学负责,我们这个基于B/S的系统可以让所有标注人员在同一个平台上看到别的同学的标注结果,实时性相对较高。
	\item 不同标注数据之间有约束关系,每项数据本身也需要满足一些条件。例如同一个句子的分词和词性个数需要一样,词性应该是有效的。该系统为各种约束条件提供了检查,保证大多数容易出现的错误能够被自动检查出来,提醒标注人员修改,防止引入标注错误。
	\item 有些标注字段有自动标注工具,例如分词工具、词性标注工具、句法成分分析器等等,对这类数据的标注通常是基于自动分析结果进行人工修正,而不是由标注人员去标注从零开始标注,这样可以降低标注的工作量,提高标注效率。另一方面,自动分析也可以用于对比人工修正结果与自动分析结果之间的差异。该系统对分词、命名实体识别、术语识别、词性标注、成分句法分析、试题语义模板提供了生成自动分析结果的功能。
	\item 提供标注快捷键、标注提示等等,尤其在试题语义模板的标注功能上,避免了标注人员还要另外参考模板体系规范,提高标注速度。
	\item 对所有的标注数据记录标注人信息,数据出错或者有异议时,可以方便找到原始标注人进行讨论和修改。
	\item 支持试题检索,可以在系统内所有标注数据中快速找到包含相关关键字的试题。
	\item 支持按所属试卷、按试题语义模板类型两种方式导出数据,并且可以选择导出哪些标注内容,方便各项任务的研究人员获取数据。
	\item 在添加试题时支持查重功能,对系统中已有的试题和试卷进行提示,避免重复劳动。
\end{enumerate}

\section{系统架构}
该系统采用了B/S架构,使用python的django框架进行WEB环境的搭建,在后台使用了mongodb数据库来存储数据。系统架构如图\ref{django}所示。

\begin {figure}[!htbp]
	\centering
	\includegraphics[width=0.4\linewidth]{{django}.jpeg}
	\caption{试题标注系统基于django的架构}
	\label{django}
\end{figure}

系统主要包含四个核心功能:试题上传、试题浏览、试题标注、试题导出。对选择题和主观题分开管理,但是提供基本相同的标注流程,并对两种试题都提供了上述四种核心功能,同时对两者的差异进行了处理。

在试题标注中,由于各项标注内容之间存在相互依赖关系,系统中为了这种依赖关系,要求在标注某一项数据之前,需要完成其依赖的标注内容在该试题上的标注。选择题和主观题的标注流程如图\ref{tagdep}所示。

首先是试题导入,试题为特定格式的文本文件,选择题和主观题分别在通过不同的文件上传。系统会判断txt文件内部格式是否正确,例如题目的层次、选项的个数等等,经过解析的试题数据会以层次化的结构,以一个document的形式存储起来。对于选择题,会特别进行选项并列拆分的标注,为第\ref{chapter:split}章中的地理选择题选项拆分任务提供标注数据。

\begin {figure}[!htbp]
\centering
\includegraphics[width=\linewidth]{{tagdep}.png}
\caption{标注流程示意图}
\label{tagdep}
\end{figure}


图中标注任务之间的有向箭头表示数据之间的依赖关系,只有完成某一项任务的标注后,从该项任务指出的箭头所指向的任务才可以进行标注。具体来说,试卷会首先进行分词标注,完成后可进行试题语义模板(图中简称“模板”)、实体和术语的标注;完成实体和术语的标注后可以进行词性标注;完成词性标注后可以进行成分句法分析的标注。此外还有一些解题相关任务的数据标注未在该图中体现,例如套话、上下文、题干核心成分分类、选项类别、题干前导部分分类等,这些标注仅依赖分词标注。

图中上方指向标注任务的箭头,连接了标注任务和该任务在本系统中对应使用的自动标注工具的名字。nlpTool是项目组其他成员基于地理试题领域文本进行性能调优的自动分析工具,支持分词、实体(时间、地点、数量词)、术语、词性等四项标注,例如在分词任务中加入了地理领域词典,在术语识别中使用地理术语表等资源。Berkeley Parser用来给试题文本生成成分句法分析结果。

图中下方的Template ConfigFile记录了目前系统支持的模板类型及其填槽规范和示例,在标注的过程中,可能发现某个模板设计地不合理,或是想要增删一些模板,甚至更改整个模板体系,通过这个配置文件,可以灵活地修改系统支持的模板。

\section{功能说明及使用方法}
系统提供的标注是基于试卷的,通过试卷名找到对应试卷,然后开始标注。支持两种标注方式,一种是按照标注内容一项一项对整套试卷进行标注,另一种是对某一道题进行所有标注。通常在使用过程中,由于标注人员各有分工,基本上是采用按内容标注的方式,在系统中我们称之为“单项标注”,词性的单项标注如图\ref{tagpos}所示。以题目为单位的标注通常用于标注完成后对某道题的标注进行修改,我们称之为“单句标注”,如图\ref{tag_single}所示。

\begin {figure}[!htbp]
\centering
\includegraphics[width=\linewidth]{{tagpos}.png}
\caption{词性单项标注页面}
\label{tagpos}
\end{figure}

可以以试卷为单位整体标注的数据字段仅有分词、时间地点(即实体,实际上还包括数量词的标注)、词性、术语、套话/上下文标注这5项内容。因为这些标注对于每一个试题文本来说都很简短,两行文本或者几个填空就可以完成,并且同一道选择题的四个(或以上)试题文本之间常常互有关联,例如对于分词来说,多个选项之间的题面分词结果是一致的。所以系统提供了在同一个页面中对该试卷所有试题文本进行某项数据的标注功能。

对于不方便进行整体标注的数据,我们为每个试题文本单独显示一个标注页面,例如后面介绍的试题文本拆分标注、AMR标注都是每次标注一个试题文本。

\begin {figure}[!htbp]
	\centering
	\includegraphics[width=\linewidth]{{tag_single}.png}
	\caption{单句标注页面}
	\label{tag_single}
\end{figure}

\subsection{基本使用流程}
本节以选择题为例,简要说明一张试卷的主要标注流程。首先在浏览试卷的页面会显示系统中所有的选择题试卷,并显示每一项内容的标注完成情况及标注人信息,提供进入标注的超链接。此外,该页面还提供检索功能:对试卷名关键字的检索、对各项数据标注状态的检索。该页面如图\ref{browse}所示。

\begin {figure}
	\centering
	\includegraphics[width=\linewidth]{{browse}.jpg}
	\caption{试题浏览页面}
	\label{browse}
\end{figure}

如果我们打算标注某个试卷的某项数据,需要点开图\ref{browse}所示页面中试卷信息所在行的最后一列中“单项标注”链接,系统会根据该试卷当前的标注情况,为所有可以进行标注的内容提供超链接。可以标注的内容为:其所依赖的标注数据已经全部标注完成、依赖该标注的标注数据还未提交。不满足这两个条件的标注内容不能再通过“单项标注”进行修改,只能通过“单句标注”修改该数据及与之相关的依赖数据。例如图\ref{browse}中的第一份试卷“2016年-北京市西城区-一模”,在它的“单项标注”页面中,提供如图\ref{singleitem}的标注超链接。因为分词是时间地点标注的依赖数据,当时间地点标注完成后,分词不能再整体标注。时间地点是词性标注的依赖数据,所以这里时间地点不能再整体标注。后五项数据都没有被依赖数据,而它们的依赖数据分词已经完成标注,所以这些内容是当前可整体标注的。

\begin {figure}
	\centering
	\includegraphics[width=\linewidth]{{single_item}.jpeg}
	\caption{单项标注页面}
	\label{singleitem}
\end{figure}

\subsection{试题文本拆分标注}
\begin {figure}[!htbp]
\centering
\includegraphics[width=0.7\linewidth]{{split_tag}.jpeg}
\caption{拆分标注页面}
\label{split_tag}
\end{figure}

系统会对所有选项中含有逗号的试题文本提供一个标注界面,并忽略不含逗号的试题文本,如图\ref{split_tag}所示。我们需要通过图\ref{browse}中试卷所在行的最后一类中的“标注拆分”超链接进入这个界面,这个超链接只会在还未完成标注拆分或已完成标注拆分但没有提交分词结果(提交分词结果后不能再进入标注拆分)的试卷中显示,所以该图中没有这种试卷也就没有这个超链接。

首先我们可以选择该试题文本是否可拆分,如果选择“是”,则会显示下面的拆分后的几个部分。如果公共部分右边界位于选项中而不是题面和选项的边界处,则第二个部分及后面的部分,需要在最前面补上缺失的公共部分。例如在例图中,选项文本的拆分情况为“下列关于太阳方向的叙述,不正确的为@冬至日某地/太阳升起的方位是东偏南,落下的方位是西偏南”,原本两个逗号隔开的部分是“冬至日某地太阳升起的方位是东偏南”及“落下的方位是西偏南”,但是由于公共部分右边界位于“冬至日某地”后面,所以会将‘@’和‘/’之间的部分补全到第二部分的最前面,这样得到的每一个部分和题面进行拼接,就得到了拆分后的多个句子。如果选择“否”,则不会显示拆分后的部分。这样就标注了是否可拆分,以及可拆分的情况下公共部分右边界的位置。

\subsection{AMR标注}
中文AMR的标注已经有一个更加专业和完整的工具,李斌老师开发了这个工具并用于中文语料的标注\cite{Li2016Annotating} ,但不方便直接接入我们的系统。为了避免重复劳动,我们没有重新开发一套完整的AMR标注工具,包括对中文Propbank的支持等功能。在地理试题标注系统中加入AMR标注的目的主要是为了保持数据的一致性和标注的完整性,将AMR标注结果和其他标注结果保存在一起,便于检索和使用。

系统为AMR标注提供了基本的功能,提供了一个简单的输入框,可以将使用专业工具标注得到的结果添加进来保存入系统,同时这个标注结果是随时可以修改的。标注界面如图\ref{tagamr}所示。

\begin {figure}
\centering
\includegraphics[width=0.9\linewidth]{{tagamr}.jpg}
\caption{AMR标注页面}
\label{tagamr}
\end{figure}

\subsection{标注数据导出}
我们提供两种数据导出方式:一种是以试卷为单位,每份试卷的数据生成一个文件;一种是以试题语义模板的类型为单位,找出所有对应类型的试题文本,为同一类型的试题文本及其标注数据生成一个文件。在两种导出方式下,都可以根据使用者的需要,选择下载哪些标注数据。例如按模板类型导出的页面如图\ref{extract}所示。

\begin {figure}
	\centering
	\includegraphics[width=0.9\linewidth]{{extract}.jpg}
	\caption{按模板类型导出数据的页面}
	\label{extract}
\end{figure}

\section{本章小结}
本章介绍了地理试题标注系统的主要功能,包括整个系统的架构,基本的使用流程等等。该系统可以为多项地理试题问答系统的自然语言处理相关任务提供高质量的语料,包括本文对试题理解所做的复杂试题拆分、AMR语义分析等。这个系统使标注人员之间的合作更加便捷,对标注效率的提升也很明显,并且可以检测出很多人工标注中可能出现的错误。

系统主要分为试题上传、检索、标注、导出等四个功能,已经投入实际使用,为十几位标注人员提供自然语言处理各项任务及地理试题理解相关的各项任务的标注环境,目前已经对79份试卷进行了标注。在使用过程中,也接受了多位标注人员的反馈,对系统的便捷性、实用性、健壮性进行了提升。

\chapter{总结与展望}
\section{工作总结}
本文工作的背景是高考地理自动问答系统,本文工作的切入点主要是对高考地理试题的理解。试题理解是多方面的,包括基本的NLP各项任务,例如命名实体识别、时间识别、句法分析等等,也包括对句子语义的理解;此外,在项目中还提出了试题语义模板的概念,每一个题目的文本都需要转换成这样的模板格式,以便于后续的推理。地理试题的理解具有领域特点,本文的工作从几个角度出发,从不同方面来加强对地理试题的理解。

本文从以下几方面展开了具体工作:
\begin{enumerate}
  \item 针对地理选择题的特点,提出了复杂选择题试题文本的拆分方法,即对选项中含有逗号的较长选项,判断其是否描述了两件或以上可以独立判断的事情,如果是,再去寻找这两个子句或者短语的公共句子前缀,将句子前缀与这几个部分分别拼接,得到几个较短的简单句子。在第一步判断是否可拆分中,使用了最大熵模型作为分类器,并对一系列上下文特征的效果进行验证;在第二步中,在对数据特点进行观察后,提出了一种基于规则的启发式方法,来寻找公共部分右边界。这个方法侧重于在尽可能高的不可拆分数据的召回率下,使可拆分数据的召回率达到一定水平。应用后,可以使一部分长句简化为短句进行后续处理,进而简化了试题语义模板的转换等工作在这些试题文本上的处理。这项工作从句子结构出发,间接地对试题语义理解的工作做出提升。
  \item 将AMR的自动解析算法JAMR修改应用于中文,并在中英文共五个语料(包括一个小规模的地理选择题语料)上进行了测试。实验结果表明,JAMR对中文的AMR解析效果明显差于英文,主要是在关系识别一步落后明显,而在概念识别阶段则无明显性能差异;此外,AMR对较短的、书面化的文本,例如地理选择题文本上,通过较少的标注数据就能够得到和大数据集差不多的实验效果,说明在地理试题领域的AMR应用还是比较有前景的。这项工作探索了用深层语义理解方法来理解地理试题,尽管目前效果不尽如人意,但是为未来进一步的工作打下了基础。
  \item 设计开发了一个地理试题标注系统,通过B/S架构可以使标注团队在同一份数据上进行合作标注,提供了包括各项基本自然语言处理任务、地理试题解题相关任务的标注,这其中就有另两项工作需要的拆分数据和AMR数据的标注。对地理领域标注数据的需求催生了本系统,这个标注系统也让标注人员可以更加高效、高质量地完成数据标注,为整个高考地理试题自动问答系统的多项任务提供了数据基础,促进了对地理试题理解的各项工作的开展。
  \end{enumerate}

\section{未来工作}
现有工作中,仍然存在一些不足之处,未来工作可以基于这些点进行更深入的研究:
\begin{enumerate}
  \item 在地理试题拆分任务上,目前的可用数据还比较少,仅有500多条选项含逗号的标注数据,未来可以标注更多的数据,减小数据过拟合的影响;还可以尝试更加有效的分类模型、寻找更好的特征等等,来提高分类精度。
  \item 在AMR方面,还有许多值得研究之处,例如如何更好利用人工对齐的数据,结合对新增概念的自动对齐方法,使人工对齐的数据真正发挥作用;对JAMR的概念识别阶段,通过一些方法对训练语料中未出现的词,根据词性等信息,规则地或用模型学习出他们可以对应的概念。
  \item 针对地理试题的特点,制订一套AMR概念、关系体系,对试题理解需要重点关注的信息增加学习权重,选择性地忽略一些短语内部关系,即考虑加大AMR概念的粒度,减小AMR图的复杂度,提高关键概念和关系的识别性能,更好地为语义模板转换工作服务。
\end{enumerate}

% 参考文献。应放在\backmatter之前。
% 推荐使用BibTeX,若不使用BibTeX时注释掉下面一句。
\nocite{*}
\bibliography{sample}
% 不使用 BibTeX
%\begin{thebibliography}{2}
%
%\bibitem{deng:01a}
%{邓建松,彭冉冉,陈长松}.
%\newblock {\em \LaTeXe{}科技排版指南}.
%\newblock 科学出版社,书号:7-03-009239-2/TP.1516, 北京, 2001.
%
%\bibitem{wang:00a}
%王磊.
%\newblock {\em \LaTeXe{}插图指南}.
%\newblock 2000.
%\end{thebibliography}
\begin{acknowledgement}
  时光飞逝,转眼间在南京大学的三年硕士生活即将结束,我也将迎来新的工作和生活。在南京大学计算机科学与技术系以及自然语言处理研究组的三年,我认识了很多优秀的师长和同学,他们给我的工作和生活提供了许多直接帮助,让我受益匪浅,也通过他们自身的勤勉和严谨,给我留下了深刻的印象并树立了良好的榜样,让我间接地感受到了充满活力的学习氛围。

  首先要感谢实验室的陈家骏老师,三年前接收我进入自然语言处理实验室,让我有机会接触到这么多优秀的同学。并且陈老师和蔼可亲、治学严谨,给我在为人处世和学习工作上都带来了深刻的影响。
  
  其次,要感谢我的导师戴新宇老师,戴老师在我的三年研究生期间,一直指导我的工作,给予了我很大的帮助,在生活上也给予了很多关心。此次硕士毕业论文,也是在戴老师的指导下完成的。戴老师提出了很多宝贵的意见,使我能够顺利完成论文的写作。十分感谢戴老师的付出,让我能够走进自然语言处理的世界,使我受益良多。

  同时,也要感谢黄书剑,尹存燕,沈思,李斌,张建兵等实验室的老师们。他们在我的工作中给予了很大的帮助和有价值的建议,使我能够在学习工作中走得更好。特别是感谢李斌老师对本文工作的支持,为我们的实验提供了宝贵的数据,以及在实验过程中提出了很多宝贵的建议。此外,还感谢程川,黄家君,胡光能,牛力强,周逸初,程善伯等几位师兄,给我在学习和工作上提供了很多帮助和建议,让我感受到了实验室作为一个集体,大家相互关心和帮助的氛围,希望你们在今后的人生中越走越好!还要感谢同一级的尚迪,郁振庭,周启元,季红洁,李小婉,王韶杰几位同学,为我提供了很多工作上的帮助。另外,还特别感谢李泽宇、娄超两位师弟,在高考地理试题解答项目中,我们共同合作,互相讨论学习,两位师弟工作认真,十分感谢他们的付出。
  
  最后,我要感谢我的父母,他们给予了对我生活、学习、人生规划上无条件的支持和理解,让我能够选择我喜欢的工作和生活方式,并且在我人生的一路上给予鼓励和关心,让我感觉十分幸运。希望你们可以永远开心、幸福、健康。
   
\end{acknowledgement}
% 附录,必须放在参考文献后,backmatter前
\appendix
%\chapter{测试}
%%%%%%%%%%%%%%%%%%%%%%%%%%%%%%%%%%%%%%%%%%%%%%%%%%%%%%%%%%%%%%%%%%%%%%%%%%%%%%%
% 书籍附件
\backmatter
%%%%%%%%%%%%%%%%%%%%%%%%%%%%%%%%%%%%%%%%%%%%%%%%%%%%%%%%%%%%%%%%%%%%%%%%%%%%%%%
% 作者简历与科研成果页,应放在backmatter之后
\begin{resume}
%% 论文作者身份简介,一句话即可。
%\begin{authorinfo}
%\noindent 韦小宝,男,汉族,1985年11月出生,江苏省扬州人。
%\end{authorinfo}
%% 论文作者教育经历列表,按日期从近到远排列,不包括将要申请的学位。
%\begin{education}
%\item[20013年9月 --- 2016年6月] 南京大学计算机科学与技术系 \hfill 硕士
%\item[2009年9月 --- 2013年6月] 南京大学计算机科学与技术系 \hfill 本科
%\end{education}
\begin{systems}
\item 汤莲瑞,戴新宇. 高考地理试题标注及管理系统. 计算机软件著作权(已公开). 2016.05.23.
\end{systems}
\end{resume}

%%%%%%%%%%%%%%%%%%%%%%%%%%%%%%%%%%%%%%%%%%%%%%%%%%%%%%%%%%%%%%%%%%%%%%%%%%%%%%%
% 生成《学位论文出版授权书》页面,应放在最后一页
\makelicense

%%%%%%%%%%%%%%%%%%%%%%%%%%%%%%%%%%%%%%%%%%%%%%%%%%%%%%%%%%%%%%%%%%%%%%%%%%%%%%%
\end{document}
